%& -shell-escape -enable-write18 
\documentclass[uplatex,dvipdfmx]{jsarticle}
\usepackage[utf8]{inputenc}

\usepackage[dvipdfmx]{graphicx}
\usepackage[dvipdfmx]{color}
\usepackage[driver=dvipdfm,hmargin=19.05truemm,vmargin=25.40truemm]{geometry}
\usepackage{colortbl}
\usepackage{tcolorbox}
\usepackage{varwidth}
\usepackage{xcolor}
\PassOptionsToPackage{dvipsnames}{xcolor}

\usepackage{amsmath,amsfonts,amssymb,amsthm}
\usepackage{bm}
\usepackage[italicdiff]{physics}
\usepackage{mathtools}
\mathtoolsset{showonlyrefs=true}
\usepackage[unicode, dvipdfmx]{hyperref}
\usepackage{pxjahyper}
\hypersetup{
setpagesize=false,
    bookmarksnumbered=true,
    bookmarksopen=true,
    colorlinks=true,
    linkcolor=cyan,
    citecolor=red,
}
\usepackage{mathrsfs}
\usepackage{bussproofs}
\usepackage{enumerate}
\usepackage{pxrubrica}
\usepackage{tipa}
\usepackage{ascmac}
\usepackage{caption}
\usepackage[subrefformat=parens]{subcaption}
\usepackage{listings,jvlisting}
\usepackage{tikz}
\usetikzlibrary{math,patterns,intersections,calc,arrows,graphs}
\usepackage{float}
\usepackage{xparse}
\usepackage{url}
\usepackage{fancyhdr}
\usepackage{multicol}
\usepackage{siunitx}
\usepackage{hhline}
\usepackage{bxbase}
\usepackage[mark=***]{sectionbreak}
\usepackage[geometry]{ifsym}
\usepackage[prefernoncjk]{pxcjkcat}
\usepackage[LGR,T2A,T3,T1]{fontenc}
\usepackage[greek,latin,english,russian,japanese]{pxbabel}

\allowdisplaybreaks


\theoremstyle{definition}
\newtheorem{theorem}{Thm}
\newtheorem{corollary}{Col}
\newtheorem{lemma}{Lem}
\newtheorem{definition}{Def}
\newtheorem{proposition}{Prop}

\tcbuselibrary{most}

\renewcommand{\labelitemi}{$\circ$}
\renewcommand{\labelitemii}{$\triangleright$}

\ProvideDocumentCommand\floor{m}{\lfloor {#1} \rfloor}
\ProvideDocumentCommand{\where}{}{\mathrel{}\middle|\mathrel{}}
\ProvideDocumentCommand{\when}{m}{\quad({#1})}
\ProvideDocumentCommand{\adjoint}{}{\mathbf{\ast}}
\ProvideDocumentCommand{\conjugation}{}{\mathbf{\ast}}
\NewDocumentCommand{\Jacobi}{m m}{\frac{\partial ({#1})}{\partial ({#2})}}
\NewDocumentCommand{\JacobiPolar}{O{x,y}}{\frac{\partial ({#1})}{\partial (r,\theta)}}
\DeclareMathOperator{\Ker}{Ker}
\DeclareMathOperator{\Img}{Im}
\DeclareMathOperator{\res}{Res}
\DeclareMathOperator{\Dom}{Dom}
\DeclareMathOperator{\Ran}{Ran}

\DeclareMathOperator{\exd}{d}

\DeclareFontShape{JY2}{mc}{m}{it}{<->ssub*mc/m/n}{}
\DeclareFontShape{JY2}{mc}{m}{sl}{<->ssub*mc/m/n}{}
\DeclareFontShape{JY2}{mc}{m}{sc}{<->ssub*mc/m/n}{}
\DeclareFontShape{JY2}{gt}{m}{it}{<->ssub*gt/m/n}{}
\DeclareFontShape{JY2}{gt}{m}{sl}{<->ssub*gt/m/n}{}
\DeclareFontShape{JY2}{mc}{b}{it}{<->ssub*mc/bx/it}{}
\DeclareFontShape{JY2}{mc}{bx}{it}{<->ssub*gt/m/n}{}
\DeclareFontShape{JY2}{mc}{bx}{sl}{<->ssub*gt/m/n}{}
\DeclareFontShape{JT2}{mc}{m}{it}{<->ssub*mc/m/n}{}
\DeclareFontShape{JT2}{mc}{m}{sl}{<->ssub*mc/m/n}{}
\DeclareFontShape{JT2}{mc}{m}{sc}{<->ssub*mc/m/n}{}
\DeclareFontShape{JT2}{gt}{m}{it}{<->ssub*gt/m/n}{}
\DeclareFontShape{JT2}{gt}{m}{sl}{<->ssub*gt/m/n}{}
\DeclareFontShape{JT2}{mc}{b}{it}{<->ssub*gt/b/n}{}
\DeclareFontShape{JT2}{mc}{bx}{it}{<->ssub*gt/m/n}{}
\DeclareFontShape{JT2}{mc}{bx}{sl}{<->ssub*gt/m/n}{}

\DeclareFontShape{J30}{mc}{b}{it}{<->ssub*mc/bx/it}{}
\DeclareFontShape{J30}{mc}{b}{n}{<->ssub*mc/bx/n}{}
\DeclareFontShape{J20}{mc}{b}{it}{<->ssub*mc/bx/it}{}
\DeclareFontShape{J20}{mc}{b}{n}{<->ssub*mc/bx/n}{}
\ProvideDocumentCommand{\titleof}{m}{

\title{ここに講義タイトルを入力 \\ {#1}レポート課題}
\author{ここに名前を入力}
\date{}
}

\titleof{12/21}

\pagestyle{fancy}
\rhead{第11週課題-biseki1221.pdf}

\begin{document}

\maketitle

\begin{enumerate}[(1)]
    \item \begin{enumerate}[(i)]
        \item \begin{align}
            &C_1:
            \begin{cases}
                x=\cos t\\
                y=\sin t
            \end{cases}
            ,\frac{3}{4}\pi\le t\le \frac{7}{4}\pi\\
            &C_2:
            \begin{cases}
                x=- t\\
                y= t
            \end{cases}
            ,-\frac{1}{\sqrt{2}}\le t\le \frac{1}{\sqrt{2}}
        \end{align}
        とすると,$C_1+C_2$は$\partial \Omega$と同じ向きの区分的$C^1$-級曲線である.
        線積分の定義より
        \begin{align}
            I
            &=\int_{\partial \Omega} y \dd{x}+x^2\dd{y}\\
            &=\int_{C_1} y \dd{x}+x^2\dd{y}+\int_{C_2} y \dd{x}+x^2\dd{y}\\
            &=\int^{\frac{7}{4}\pi}_{\frac{3}{4}\pi} (-\sin^2t+\cos^3t)\dd{t}+\int^\frac{1}{\sqrt{2}}_{-\frac{1}{\sqrt{2}}} (t\cdot(-1)+t^2)\dd{t}\\
            &=-\frac{\pi}{2}+\eval[\frac{3}{4}\sin t+\frac{1}{12}\sin 3t|^{\frac{7}{4}\pi}_{t\coloneqq \frac{3}{4}\pi} +2\eval[\frac{t^3}{3}|^\frac{1}{\sqrt{2}}_{t\coloneqq 0}\\
            &=-\frac{\pi}{2}+\frac{1}{\sqrt{2}}\qty(\frac{3}{4}\cdot(-1-1)+\frac{1}{12}\cdot(-1-1))+2\cdot\frac{1}{3}\cdot\frac{1}{2\sqrt{2}}\\
            &=-\frac{\pi}{2}-\frac{5}{6}\sqrt{2}+\frac{1}{6}\sqrt{2}\\
            \therefore
            I
            &=-\frac{\pi}{2}-\frac{2}{3}\sqrt{2}
        \end{align}
        である.
        \item Greenの定理により,$I$は次の$\Omega$上の重積分と等しい:
        \begin{align}
            I
            &=\iint_\Omega (2x-1)\dd{x}\dd{y}\\
            \intertext{ほぼ全単射な写像によって極座標に変換すると}
            &=\iint_{\substack{0\le r\le 1\\\frac{3}{4}\pi\le \theta\le \frac{7}{4}\pi}} (2r\cos\theta-1)\JacobiPolar\dd{r}\dd{\theta}\\
            &=\int^{\frac{7}{4}\pi}_{\frac{3}{4}\pi} \qty(\frac{2}{3}\cos\theta-\frac{1}{2})\dd{\theta}\\
            &=\frac{2}{3}\eval[ \sin\theta |^{\frac{7}{4}\pi}_{\frac{3}{4}\pi} -\frac{\pi}{2}\\
            \therefore
            I
            &=-\frac{\pi}{2}-\frac{2}{3}\sqrt{2}
        \end{align}
        と計算することができ,値は(i)のものと一致する.
        \item 微分1-形式
        \begin{align}
            \omega = g(x,y) \dd{x} + ye^{x^2}\dd{y}
        \end{align}
        を考える.$\exd:\Omega^1(\mathbb{R}^2)\to\Omega^2(\mathbb{R}^2)$を外微分とすると,$\omega$が閉形式,すなわち$\exd(\omega) = 0$となるとき,$g(x,y)$が$\mathbb{R}^2$上で定義されればPoincaréの補題により$\omega$は完全形式である.

        \begin{align}
            g(x,y)=xy^2e^{x^2}
        \end{align}
        のとき,
        \begin{align}
            \exd(\omega)
            &=\qty(\pdv{x} (ye^{x^2})-\pdv{y}(g(x,y)) )\dd{x}\wedge\dd{y}\\
            &=\qty(2xye^{x^2}-2xye^{x^2} )\dd{x}\wedge\dd{y}\\
            &=0
        \end{align}
        となるから,
        \begin{align}
            J
            &=\int_{\partial \Omega}\omega\\
            &=0
        \end{align}
        となる.
        したがって,$J=0$となる関数$g(x,y)$として$g(x,y)=xy^2e^{x^2}$がある.
    \end{enumerate}
    \item 
    \begin{enumerate}[(i)]
        \item \begin{itemize}
            \item $C$について
            \begin{align}
                C:\begin{cases}
                    x=r\cos t\\
                    y=r\sin t
                \end{cases}
                ,0\le t \le 2\pi
            \end{align}
            というパラメタで積分する.
            \begin{align}
                \int_C\omega
                &=\int_0^{2\pi} \qty(-\frac{1}{r}\sin t\cdot(-r\sin t)+\frac{1}{r}\cos t\cdot (r\cos t))\dd{t}\\
                &=\int_0^{2\pi} \dd{t}\\
                &=2\pi
            \end{align}
            \item $S$について\\
            $S$としてどのようなものを選んでも$\displaystyle\int_S \omega$の値は等しいことは後の補題\ref{lem:1}で示すことにして,ここでは$S$として次の図に示すようなものを選ぶ.
            \begin{figure}[H]
                \centering
                \begin{tikzpicture}[domain=1:1, yscale=1, xscale=1]
                    
                    \draw[->] (-2.4,0) -- (2.5,0) node[below right] {$x$};
                    \draw[->] (0,-2.4) -- (0,2.5) node[above left] {$y$};
                    
                    \draw (0,0) node[below left] {$\mathrm{O}$};
                    \draw (1,0) node[below right] {$r$};
                    \draw (0,1) node[above left] {$r$};
                    
                    \draw[color=red] (1,1) -- (-1,1) -- (-1,-1) -- (1,-1) -- cycle node[above right] {$S$};

                \end{tikzpicture}
                \caption{正方形$S$}
                \label{fig:Squ}
            \end{figure}
            $S$のパラメタ付けとして,
            \begin{align}
                S_1&:\begin{cases}
                    x=-rt\\
                    y=r
                \end{cases}
                ,-1\le t \le 1\\
                S_2&:\begin{cases}
                    x=-r\\
                    y=-rt
                \end{cases}
                ,-1\le t \le 1\\
                S_3&:\begin{cases}
                    x=rt\\
                    y=-r
                \end{cases}
                ,-1\le t \le 1\\
                S_4&:\begin{cases}
                    x=r\\
                    y=rt
                \end{cases}
                ,-1\le t \le 1
            \end{align}
            を考えると
            \begin{align}
                \int_S\omega
                &=\int_{S_1}\omega+\int_{S_2}\omega+\int_{S_3}\omega+\int_{S_4}\omega\\
                &=\int_{-1}^1\frac{-r\cdot(-r)+(-r)\cdot(-r)-(-r)\cdot r+r\cdot r}{r^2(1+t^2)}\dd{t}\\
                &=8\int_0^1\frac{1}{(1+t^2)}\dd{t}\\
                &=8\arctan 1\\
                &=2\pi
            \end{align}
            である.
        \end{itemize}
        \item
        \begin{itemize}
            \item $T$について\\
            $T\not\ni(0,0)$であるから,$T$上の点から原点の距離の下限$m$が存在して$m>0$を満たす.
            
            $r\coloneqq \frac{m}{2}$としたときの$C$を$C_{\frac{m}{2}}$とおき,$C_{\frac{m}{2}}$と$T$で囲まれる領域を$D$とする.
            
            定義から$D\subset U$である.Greenの定理より
            \begin{align}
                \int_T \omega - \int_{C_{\frac{m}{2}}} \omega 
                &=\iint_D \qty(\pdv{x}(\frac{x}{x^2+y^2})-\pdv{y}(\frac{-y}{x^2+y^2}))\dd{x}\dd{y}\\
                &=\iint_D \qty(\frac{-x^2+y^2}{(x^2+y^2)^2}-\frac{-x^2+y^2}{(x^2+y^2)^2})\dd{x}\dd{y}\\
                &=0\\
                \therefore
                \int_T \omega 
                &= \int_{C_{\frac{m}{2}}}\omega 
                = 2\pi
            \end{align}
            \item $\Gamma$について\\
            $\Gamma$は,原点のまわりを2周する閉曲線であり,適当に分割することで区分的に$C^1$-級曲線であることがわかる.
            また,第4象限上$\Gamma$は2点で自身と交差している.この2点を$A,B$とすると,$\Gamma=\Gamma_{AA}+\Gamma_{AB}+\Gamma_{BB}+\Gamma_{BA}$となるようにできる.ただし,$\Gamma_{XY} (X,Y=A,B)$は$X$と$Y$を端点とし$\Gamma$と同じ向きの曲線である.

            $\Gamma_{AA}$が内側に原点がある方であるように$A$を定める.
            \begin{align}
                \int_\Gamma \omega
                &=\int_{\Gamma_{AA}} \omega+\int_{\Gamma_{AB}} \omega+\int_{\Gamma_{BB}} \omega+\int_{\Gamma_{BA}} \omega\\
                &=\int_{\Gamma_{AA}} \omega+\int_{\Gamma_{ABA}} \omega &\qq{($\because$ Greenの定理)}
            \end{align}
            である.ただし,$\Gamma_{ABA}=\Gamma_{AB}+\Gamma_{BA}$である.

            $\Gamma_{AA},\Gamma_{ABA}$はともに閉曲線である.上の$T$についての議論と同様に,それぞれの曲線が囲う領域の内部に$C$がおさまるように$r$を定めることができて,それぞれの曲線とこのようにして定めた$C$で囲まれる領域で積分した$\omega$の値は$0$であるから,Greenの定理より
            \begin{align}
                \int_{\Gamma_{AA}} \omega &= 2\pi\\
                \int_{\Gamma_{ABA}} \omega &= 2\pi
            \end{align}
            である.よって
            \begin{align}
                \int_\Gamma \omega
                &=\int_{\Gamma_{AA}} \omega+\int_{\Gamma_{ABA}}\\
                &=4\pi
            \end{align}
            である.
        \end{itemize}
        \item そのような$F$が存在したとする.つまり,$\omega$が完全形式だと仮定する.このとき
        \begin{align}
            \int_C \omega 
            &= \int_C \dd{F}\\
            &= F(r,0)-F(r,0) &\qq{($\because$微分積分学の基本定理)}\\
            &= 0
        \end{align}
        であるが,これは(i)で求めた$\displaystyle\int_C \omega =2\pi$と矛盾する.

        よって,そのような$F$は存在しない.\qed
        \item まず,次の補題を示す.
        \begin{lemma}
            閉である微分1-形式$\eta$と区分的に$C^1$-級である$U$上の閉曲線$K$に対して,$K$が原点のまわりを$n$回周回するとき
            \begin{align}
                \int_K \eta = n\int_C \eta
            \end{align}
            である.\label{lem:1}
        \end{lemma}
        \begin{proof}
            (ii)の$\Gamma$についての議論と同じようにして,
            \begin{align}
                \int_K \eta
                &=\sum_{i=1}^n\int_{K_i}\eta
            \end{align}
            であり,かつ各$K_i$が囲う領域に原点が含まれるように$U$上の閉曲線の列$\qty{K_i}\quad (i = 1,\cdots n)$をとることができる.

            それぞれの$K_i$は原点を通らないので原点との距離の下限が存在して,それは正である.よって$C$が$K$の内部におさまるように$r$をとることができる.
            $K_i,C$に囲まれた領域$D$に対しGreenの定理より
            \begin{align}
                \int_{K_i} \eta -  \int_C \eta 
                &=  \int_D \dd{\eta} = 0 & \text{($\because$ $\eta$は閉)}\\
                \int_{K_i} \eta 
                &= \int_C \eta
                \intertext{よって}
                \int_K \eta &= n\int_C \eta
            \end{align}
            を得る.
        \end{proof}
        % \newpage
        示すものは次である:
        \begin{proposition}
            閉である微分1-形式$\tau$に対して,スカラー$\alpha \in \mathbb{R}$が存在して$\eta=\tau-\alpha \omega$は完全形式である.
        \end{proposition}
        \begin{proof}
            \begin{align}
                a=\int_C\tau
            \end{align}
            とし,
            \begin{align}
                \alpha = \frac{a}{2\pi}
            \end{align}
            と定める.このとき,区分的に$C^1$-級である$U$上の任意の閉曲線$K$に対して
            \begin{align}
                \int_K \eta
                &=\qty(\int_K \tau)- \alpha\qty(\int_K \omega)\\
                &=n\qty(a- \frac{a}{2\pi} \cdot 2\pi) &\text{($\because$補題\ref{lem:1})}\\
                &=0 \label{eq:circ_0}
            \end{align}
            が成り立つ.ただし,$n$は原点のまわりを周回した回数である.
            \begin{align}
                U_1 &= U \setminus \qty{(x,y) \where x\le 0 \land y=0}\\
                U_2 &= U \setminus \qty{(x,y) \where x=0 \land y\le 0}
            \end{align}
            とすると,$U_1, U_2$は星状領域である.なぜなら,$A(-1,0), P \in U_1$を結ぶ線分上の点は$U_1$に属するからである.$U_2$についても同様.

            \begin{figure}[H]
                \centering
                \begin{tikzpicture}[domain=1:1, yscale=0.6, xscale=0.6]
                    
                    % \draw[very thin, dashed,color=gray] (-2.3,-2.3) grid (2.3,2.3);
        
                    \draw[->] (-2.4,0) -- (2.5,0) node[below right] {$x$};
                    \draw[->] (0,-2.4) -- (0,2.5) node[above left] {$y$};
                    
                    \draw (0,0) node[below left] {$\mathrm{O}$};
                    \draw[color=red] (-1.7,0) node[below left] {$A(-1,0)$} -- (1.2,1) node[above right] {$P$};
                    
                \end{tikzpicture}
                \caption{$U_1$は星状領域である}
            \end{figure}
            
            $X$と$Y$を結ぶ線分で,向きが$X$から$Y$である曲線を$\overrightarrow{XY}$と書くことにする.
            $A(-1,0)\in U_1,B(0,-1)\in U_2, P\in U$に対して,
            \begin{align}
                G(P)&\coloneqq
                \begin{dcases}
                    G_1(P)=\int_{\overrightarrow{AP}}\eta & \text{($P\in U_1$のとき)}\\
                    G_2(P)=\int_{\overrightarrow{AB}}\eta + \int_{\overrightarrow{BP}}\eta & \text{($P\in U \setminus U_1$のとき)}\\
                \end{dcases}
            \end{align}
            と定める$G$は$C^\infty$-級であることを示す.
            \begin{lemma}
                $G_1$(resp. $G_2$)は$U_1$(resp. $U_2$)上で$C^\infty$-級である.\label{lem:g1u1}
            \end{lemma}
            \begin{proof}
                $P(a,b)\in U_1$を任意にとる.
                $U_1$は開集合であるから,$P$の近傍$D$が存在して$D\subset U_1$である.$\varepsilon$を$X(a + \varepsilon,b),Y(a,b + \varepsilon) \in D$となるようにとる.
                
                $\eta = \phi\dd{x}+\gamma\dd{y}$とおくと,$\phi(x,y)=u(x,y)-\alpha\cdot \dfrac{-y}{x^2+y^2}, \gamma(x,y)=v(x,y)-\alpha\cdot \dfrac{x}{x^2+y^2}$はともに$C^\infty$-級である.
                \begin{align}
                    \sigma \cdot 0 =\sigma \cdot \int_{\partial (\triangle APX)}\dd{\eta}
                    &=\int_{\overrightarrow{AP}}\eta + \int_{\overrightarrow{PX}}\eta + \int_{\overrightarrow{XA}}\eta & \text{($\because$ 式(\ref{eq:circ_0}))}\\
                    \intertext{ただし,$\sigma=\pm 1$は$\triangle APX$の周の向きによって定まる量である.}
                    \int_{\overrightarrow{AX}}\eta - \int_{\overrightarrow{AP}}\eta &=\int_{\overrightarrow{PX}}\eta \\
                    G_1(a + \varepsilon,b) - G_1(a,b) &=\int_a^{a+\varepsilon} P(t,b)\dd{t}\\
                    \intertext{$x=a+\varepsilon$とおいて}
                    G_1(x,b) - G_1(a,b) &=\int_a^{x} \phi(t,b)\dd{t}\\
                    {G_1}_x(x,b) &=\phi(x,b)
                    \intertext{$y$についても同様に,}
                    {G_1}_y(a,y) &=\gamma(a,y)
                \end{align}
                が成立.$P$は任意であったから,$G_1$は$U_1$上で$C^\infty$-級である.
                
                同様の議論を$U_2$上$G_2$にもして,$G_2$は$U_2$上で$C^\infty$-級であることが示される.
            \end{proof}
            \begin{lemma}
                $U_1 \cap U_2$上で,$G_1=G_2$である.\label{lem:u1u2}
            \end{lemma}
            \begin{proof}
                $P\in U_1 \cap U_2$に対し,線分$AP$,線分$BP$上の各点はそれぞれ$U_1,U_2$に含まれる.三角形$ABP$とその周に対してGreenの定理を適用して,
                \begin{align}
                    \sigma \cdot 0 =\sigma \cdot \int_{\partial (\triangle ABP)}\dd{\eta}
                    &=\int_{\overrightarrow{AB}}\eta + \int_{\overrightarrow{BP}}\eta + \int_{\overrightarrow{PA}}\eta & \text{($\because$ \eqref{eq:circ_0})}\\
                    \intertext{ただし,$\sigma=\pm 1$は$\triangle ABP$の周の向きによって定まる量である.}
                    \int_{\overrightarrow{AP}}\eta&=\int_{\overrightarrow{AB}}\eta + \int_{\overrightarrow{BP}}\eta \\
                    \therefore 
                    G_1(P)&=G_2(P)
                \end{align}
                
                \begin{figure}[H]
                    \centering
                    \begin{minipage}{0.4\hsize}
                        \begin{tikzpicture}[domain=1:1, yscale=0.8, xscale=0.8]
                            
                            \draw[->] (-2.4,0) -- (2.5,0) node[below right] {$x$};
                            \draw[->] (0,-2.4) -- (0,2.5) node[above left] {$y$};
                            
                            \draw (0,0) node[below left] {$\mathrm{O}$};
                            \draw[color=orange] (-1.7,0) node[below left] {$A(-1,0)$} -- (-0.6,1.4) node[above] {$P$} -- (0,-1.7) node[below right] {$B(0,-1)$} -- cycle;
                            
                        \end{tikzpicture}
                        \caption{$\partial(\triangle ABP)\subset U$}
                    \end{minipage}
                    \centering
                    \begin{minipage}{0.4\hsize}
                        \centering
                        \begin{tikzpicture}[domain=1:1, yscale=0.8, xscale=0.8]
                            
                            \draw[->] (-2.4,0) -- (2.5,0) node[below right] {$x$};
                            \draw[->] (0,-2.4) -- (0,2.5) node[above left] {$y$};
                            
                            \draw (0,0) node[below left] {$\mathrm{O}$};
                            \draw[color=cyan] (-1.7,0) node[below left] {$A(-1,0)$} -- (1.4,1.1) node[above] {$P$} -- (0,-1.7) node[below right] {$B(0,-1)$} -- cycle;
                            
                        \end{tikzpicture}
                        \caption{$\partial(\triangle ABP)\subset U$}
                    \end{minipage}
                \end{figure}
            \end{proof}

            補題\ref{lem:u1u2}より,$U_1 \cap U_2$上で$G=G_1=G_2$である.また,
            $G$の定義より$(U \setminus U_1)(\subset U_2) $上で$G=G_2$である.
            よって,$U_2$上で$G=G_2$である.

            補題\ref{lem:g1u1}と$G$の定義より,$U_1$上で$G=G_1$は$C^\infty$-級である.また,$U_2$上で$G=G_2$は$C^\infty$-級である.
            
            以上から,
            \begin{align}
                \alpha = \dfrac{\int_C\tau}{2\pi}, \eta=\tau-\alpha \omega
            \end{align}
            のとき,
            \begin{align}
                G(P)&\coloneqq
                \begin{dcases}
                    G_1(P)=\int_{\overrightarrow{AP}}\eta & \text{($P\in U_1$のとき)}\\
                    G_2(P)=\int_{\overrightarrow{AB}}\eta + \int_{\overrightarrow{BP}}\eta & \text{($P\in U \setminus U_1$のとき)}\\
                \end{dcases}
            \end{align}
            上のように定めた$G$は$C^\infty$-級である.
        \end{proof}
    \end{enumerate}
\end{enumerate}

\end{document}