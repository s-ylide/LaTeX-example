%& -shell-escape -enable-write18 
\documentclass[uplatex,dvipdfmx]{jsarticle}
\usepackage[utf8]{inputenc}

\usepackage[dvipdfmx]{graphicx}
\usepackage[dvipdfmx]{color}
\usepackage[driver=dvipdfm,hmargin=19.05truemm,vmargin=25.40truemm]{geometry}
\usepackage{colortbl}
\usepackage{tcolorbox}
\usepackage{varwidth}
\usepackage{xcolor}
\PassOptionsToPackage{dvipsnames}{xcolor}

\usepackage{amsmath,amsfonts,amssymb,amsthm}
\usepackage{bm}
\usepackage[italicdiff]{physics}
\usepackage{mathtools}
\mathtoolsset{showonlyrefs=true}
\usepackage[unicode, dvipdfmx]{hyperref}
\usepackage{pxjahyper}
\hypersetup{
setpagesize=false,
    bookmarksnumbered=true,
    bookmarksopen=true,
    colorlinks=true,
    linkcolor=cyan,
    citecolor=red,
}
\usepackage{mathrsfs}
\usepackage{bussproofs}
\usepackage{enumerate}
\usepackage{pxrubrica}
\usepackage{tipa}
\usepackage{ascmac}
\usepackage{caption}
\usepackage[subrefformat=parens]{subcaption}
\usepackage{listings,jvlisting}
\usepackage{tikz}
\usetikzlibrary{math,patterns,intersections,calc,arrows,graphs}
\usepackage{float}
\usepackage{xparse}
\usepackage{url}
\usepackage{fancyhdr}
\usepackage{multicol}
\usepackage{siunitx}
\usepackage{hhline}
\usepackage{bxbase}
\usepackage[mark=***]{sectionbreak}
\usepackage[geometry]{ifsym}
\usepackage[prefernoncjk]{pxcjkcat}
\usepackage[LGR,T2A,T3,T1]{fontenc}
\usepackage[greek,latin,english,russian,japanese]{pxbabel}

\allowdisplaybreaks


\theoremstyle{definition}
\newtheorem{theorem}{Thm}
\newtheorem{corollary}{Col}
\newtheorem{lemma}{Lem}
\newtheorem{definition}{Def}
\newtheorem{proposition}{Prop}

\tcbuselibrary{most}

\renewcommand{\labelitemi}{$\circ$}
\renewcommand{\labelitemii}{$\triangleright$}

\ProvideDocumentCommand\floor{m}{\lfloor {#1} \rfloor}
\ProvideDocumentCommand{\where}{}{\mathrel{}\middle|\mathrel{}}
\ProvideDocumentCommand{\when}{m}{\quad({#1})}
\ProvideDocumentCommand{\adjoint}{}{\mathbf{\ast}}
\ProvideDocumentCommand{\conjugation}{}{\mathbf{\ast}}
\NewDocumentCommand{\Jacobi}{m m}{\frac{\partial ({#1})}{\partial ({#2})}}
\NewDocumentCommand{\JacobiPolar}{O{x,y}}{\frac{\partial ({#1})}{\partial (r,\theta)}}
\DeclareMathOperator{\Ker}{Ker}
\DeclareMathOperator{\Img}{Im}
\DeclareMathOperator{\res}{Res}
\DeclareMathOperator{\Dom}{Dom}
\DeclareMathOperator{\Ran}{Ran}

\DeclareMathOperator{\exd}{d}

\DeclareFontShape{JY2}{mc}{m}{it}{<->ssub*mc/m/n}{}
\DeclareFontShape{JY2}{mc}{m}{sl}{<->ssub*mc/m/n}{}
\DeclareFontShape{JY2}{mc}{m}{sc}{<->ssub*mc/m/n}{}
\DeclareFontShape{JY2}{gt}{m}{it}{<->ssub*gt/m/n}{}
\DeclareFontShape{JY2}{gt}{m}{sl}{<->ssub*gt/m/n}{}
\DeclareFontShape{JY2}{mc}{b}{it}{<->ssub*mc/bx/it}{}
\DeclareFontShape{JY2}{mc}{bx}{it}{<->ssub*gt/m/n}{}
\DeclareFontShape{JY2}{mc}{bx}{sl}{<->ssub*gt/m/n}{}
\DeclareFontShape{JT2}{mc}{m}{it}{<->ssub*mc/m/n}{}
\DeclareFontShape{JT2}{mc}{m}{sl}{<->ssub*mc/m/n}{}
\DeclareFontShape{JT2}{mc}{m}{sc}{<->ssub*mc/m/n}{}
\DeclareFontShape{JT2}{gt}{m}{it}{<->ssub*gt/m/n}{}
\DeclareFontShape{JT2}{gt}{m}{sl}{<->ssub*gt/m/n}{}
\DeclareFontShape{JT2}{mc}{b}{it}{<->ssub*gt/b/n}{}
\DeclareFontShape{JT2}{mc}{bx}{it}{<->ssub*gt/m/n}{}
\DeclareFontShape{JT2}{mc}{bx}{sl}{<->ssub*gt/m/n}{}

\DeclareFontShape{J30}{mc}{b}{it}{<->ssub*mc/bx/it}{}
\DeclareFontShape{J30}{mc}{b}{n}{<->ssub*mc/bx/n}{}
\DeclareFontShape{J20}{mc}{b}{it}{<->ssub*mc/bx/it}{}
\DeclareFontShape{J20}{mc}{b}{n}{<->ssub*mc/bx/n}{}
\ProvideDocumentCommand{\titleof}{m}{

\title{ここに講義タイトルを入力 \\ {#1}レポート課題}
\author{ここに名前を入力}
\date{}
}
\titleof{10/19}

\begin{document}
    
\maketitle

\begin{enumerate}[(1)]
    \item 
    \begin{align}
        F^\prime (x)&=(x^2+\cos^2 x)\sin x\\
        F^{\prime\prime} (x)&=\sin x(2x-2\sin x\cos x)+\cos x(x^2+\cos^2x)
    \end{align}
    $F^\prime (x)=0$となる$x$を求める.
    \begin{itemize}
        \item $x=n\pi\ (\exists n\in \mathbb{Z})$のとき,$F^\prime (x)=0$
        \item $x\ne n\pi\ (\forall n\in \mathbb{Z})$のとき,
        $\sin x\ne 0$であり$x\ne 0$ゆえ$x^2>0$である.\\
        よって$x^2+\cos^2x>0$で,$F^\prime (x)\ne 0$
    \end{itemize}
    極大値を与える$x$は,$x=n\pi\ (\exists n\in \mathbb{Z})$かつ
    \begin{itemize}
        \item $x=2k\pi\ (\exists k\in\mathbb{Z})$のとき\\
        \begin{equation}
            F^{\prime\prime} (x)=(x^2+1)\cdot 1>0
        \end{equation}
        \item $x=(2k+1)\pi\ (\exists k\in\mathbb{Z})$のとき\\
        \begin{equation}
            F^{\prime\prime} (x)=(x^2+1)\cdot (-1)<0
        \end{equation}
    \end{itemize}
    すなわち,極大値を与える$x$は$x=(2k+1)\pi\ (k=0,1,2,\cdots)$で,そのとき
    \begin{align}
        F(x)
        &=\int_0^{x}(t^2+\cos^2t)\sin t\dd{t}\\
        &=\eval[-t^2\cos t+2t\sin t+2\cos t-\frac{1}{3}\cos^3 t|_{t:=0}^{x}\\
        &=x^2-2+\frac{1}{3}-(2-\frac{1}{3})\\
        &=x^2-\frac{10}{3}
    \end{align}
    である.
    \item 
    \begin{enumerate}[(i)]
        \item $\displaystyle \dv{x}(\arctan x)=\dfrac{1}{1+x^2}$であるから
        \begin{align}
            \dv{x}(\arctan x)
            &=\frac{1}{a}\cdot\frac{1}{1+\frac{x^2}{a^2}}\\
            &=\frac{1}{a}\cdot\frac{a^2}{x^2+a^2}\\
            &=\frac{a}{x^2+a^2}
        \end{align}
        \item \ 
        \begin{itemize}
            \item $\displaystyle \int \frac{a+bx}{x^2+1}\dd{x}$
            \begin{align}
                \int \frac{a+bx}{x^2+1}\dd{x}
                &=a\int\frac{1}{x^2+1}\dd{x}+\frac{b}{2}\int\frac{2x}{x^2+1}\dd{x}\\
                &=a\arctan{x}+\frac{b}{2}\log(x^2+1)+C\ (C\in \mathbb{C})
            \end{align}
            \item $\displaystyle \int \frac{a+bx}{(x^2+1)^2}\dd{x}$
            \begin{lemma}
                $n>1$に対し
                \begin{equation}
                    \int \frac{1}{(x^2+1)^n}\dd{x}=
                    \frac{1}{2n-2}\frac{x}{(x^2+1)^{n-1}}+
                    \frac{2n-3}{2n-2}\int \frac{1}{(x^2+1)^{n-1}}\dd{x}
                \end{equation}
                が成り立つ.
            \end{lemma}
            \begin{proof}
                右辺を微分すると,
                \begin{align}
                    &
                    \frac{1}{2n-2}\frac{1}{(x^2+1)^{n-1}}
                    -\frac{x^2}{(x^2+1)^{n}}+
                    \frac{2n-3}{2n-2}\frac{1}{(x^2+1)^{n-1}}\\
                    =&
                    \frac{x^2+1}{(x^2+1)^{n}}
                    -\frac{x^2}{(x^2+1)^{n}}\\
                    =&
                    \frac{1}{(x^2+1)^{n}}
                \end{align}
                となり,これは左辺を微分したものと一致する.
            \end{proof}
            \begin{align}
                \int \frac{a+bx}{(x^2+1)^2}\dd{x}
                &=a\int\frac{1}{(x^2+1)^2}\dd{x}+\frac{b}{2}\int\frac{2x}{(x^2+1)^2}\dd{x}\\
                &=a\qty{\frac{x}{2(x^2+1)}+\frac{1}{2}\arctan x}+
                \frac{b}{2}\qty(-\frac{1}{x^2+1})+C& (C\in \mathbb{C})\\
                &=\frac{ax-b}{2(x^2+1)}+\frac{a}{2}\arctan x+C& (C\in \mathbb{C})
            \end{align}
            \item $\displaystyle \int \frac{x}{(x^2+x+1)(x^2+2x-3)}\dd{x}$\\
            $\displaystyle \omega=e^{\frac{2\pi}{3}i}$とおく.Heavisideの展開定理により
            \begin{align}
                &\frac{x}{(x-\omega)(x-\omega^2)(x-1)(x+3)}\\
                =&
                \frac{A}{x-\omega}+
                \frac{B}{x-\omega^2}+
                \frac{C}{x-1}+
                \frac{D}{x+3}& (A,B,C,D\in \mathbb{C})
            \end{align}
            と書けて,これは$E=A+B,F=-\omega(A\omega+B)$とすることで
            \begin{equation}
                \frac{Ex+F}{x^2+x+1}+
                \frac{C}{x-1}+
                \frac{D}{x+3} \quad (C,D,E,F\in \mathbb{C})
            \end{equation}
            と表せる.これらを求めると,
            \begin{align}
                &\frac{x}{(x^2+x+1)(x-1)(x+3)}\\
                =&
                \frac{1}{21}\frac{-4x+1}{x^2+x+1}+
                \frac{1}{12}\frac{1}{x-1}+
                \frac{3}{28}\frac{1}{x+3}\\
                =&
                -\frac{2}{21}\frac{2x+1}{x^2+x+1}+
                \frac{3}{21}\frac{1}{x^2+x+1}+
                \frac{1}{12}\frac{1}{x-1}+
                \frac{3}{28}\frac{1}{x+3}
            \end{align}
            となる.よって,
            \begin{align}
                &\int \frac{x}{(x^2+x+1)(x^2+2x-3)}\dd{x}\\
                =&
                -\frac{2}{21}\int \frac{2x+1}{x^2+x+1}\dd{x}+
                \frac{3}{21}\int \frac{1}{x^2+x+1}\dd{x}+
                \frac{1}{12}\int \frac{1}{x-1}\dd{x}+
                \frac{3}{28}\int \frac{1}{x+3}\dd{x}\\
                =&
                -\frac{2}{21}\log(x^2+x+1)+
                \frac{2\sqrt{3}}{21}\int \frac{\frac{\sqrt{3}}{2}}{(x+\frac{1}{2})^2+\frac{3}{4}}\dd{x}+
                \frac{1}{12}\int \frac{1}{x-1}\dd{x}+
                \frac{3}{28}\int \frac{1}{x+3}\dd{x}\\
                =&
                -\frac{2}{21}\log(x^2+x+1)+
                \frac{2\sqrt{3}}{21}\arctan(\frac{2x+1}{\sqrt{3}})+
                \frac{1}{12}\log\abs{x-1}+
                \frac{3}{28}\log\abs{x+3}+C& (C\in \mathbb{C})
            \end{align}
            である.
        \end{itemize}
    \end{enumerate}
    \item 
    \begin{enumerate}[(i)]
        \item \begin{align}
            1 + \tan^2 x
            &=\frac{\cos^2x}{\cos^2x}+\frac{\sin^2x}{\cos^2x}\\
            &=\frac{\cos^2x+\sin^2x}{\cos^2x}\\
            &=\frac{1}{\cos^2x}
        \end{align}
        である.
        \begin{itemize}
            \item $\sin x$
            \begin{align}
                \sin x
                &=2\sin\frac{x}{2}\cos\frac{x}{2}\\
                &=2\tan\frac{x}{2}\cos^2\frac{x}{2}\\
                &=\frac{2u}{1+u^2}
            \end{align}
            \item $\cos x$
            \begin{align}
                \cos x
                &=2\cos^2\frac{x}{2}-1\\
                &=\frac{2}{1+u^2}-1\\
                &=\frac{1-u^2}{1+u^2}
            \end{align}
            \item $\displaystyle \dv{u}{x}$
            \begin{align}
                \dv{u}{x}
                &=\frac{1}{2}\frac{1}{\cos^2\frac{x}{2}}\\
                &=\frac{1+u^2}{2}
            \end{align}
        \end{itemize}
        \item 
        \begin{align}
            \int_0^\frac{\pi}{2}\frac{1+\sin x}{(1+\cos x)^2}\dd{x}
            &=\int_0^\frac{\pi}{2}\frac{\frac{(1+u)^2}{1+u^2}}{(\frac{2}{1+u^2})^2}\dd{x}\\
            &=\int_0^1(1+u^2)(1+u)^2\frac{1}{4}\cdot\frac{2}{1+u^2}\dd{u}\\
            &=\frac{1}{2}\int_0^1(1+u)^2\dd{u}\\
            &=\frac{1}{2}\eval[\frac{1}{3}(1+u)^3|_{0}^{1}\\
            &=\frac{1}{6}(8-1)\\
            &=\frac{7}{6}
        \end{align}
    \end{enumerate}
\end{enumerate}

\end{document}