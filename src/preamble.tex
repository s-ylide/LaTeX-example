%& -shell-escape -enable-write18 
\documentclass[uplatex,dvipdfmx]{jsarticle}
\usepackage[utf8]{inputenc}

\usepackage[dvipdfmx]{graphicx}
\usepackage[dvipdfmx]{color}
\usepackage[driver=dvipdfm,hmargin=19.05truemm,vmargin=25.40truemm]{geometry}
\usepackage{colortbl}
\usepackage{tcolorbox}
\usepackage{varwidth}
\usepackage{xcolor}
\PassOptionsToPackage{dvipsnames}{xcolor}

\usepackage{amsmath,amsfonts,amssymb,amsthm}
\usepackage{bm}
\usepackage[italicdiff]{physics}
\usepackage{mathtools}
\mathtoolsset{showonlyrefs=true}
\usepackage[unicode, dvipdfmx]{hyperref}
\usepackage{pxjahyper}
\hypersetup{
setpagesize=false,
    bookmarksnumbered=true,
    bookmarksopen=true,
    colorlinks=true,
    linkcolor=cyan,
    citecolor=red,
}
\usepackage{mathrsfs}
\usepackage{bussproofs}
\usepackage{enumerate}
\usepackage{pxrubrica}
\usepackage{tipa}
\usepackage{ascmac}
\usepackage{caption}
\usepackage[subrefformat=parens]{subcaption}
\usepackage{listings,jvlisting}
\usepackage{tikz}
\usetikzlibrary{math,patterns,intersections,calc,arrows,graphs}
\usepackage{float}
\usepackage{xparse}
\usepackage{url}
\usepackage{fancyhdr}
\usepackage{multicol}
\usepackage{siunitx}
\usepackage{hhline}
\usepackage{bxbase}
\usepackage[mark=***]{sectionbreak}
\usepackage[geometry]{ifsym}
\usepackage[prefernoncjk]{pxcjkcat}
\usepackage[LGR,T2A,T3,T1]{fontenc}
\usepackage[greek,latin,english,russian,japanese]{pxbabel}

\allowdisplaybreaks


\theoremstyle{definition}
\newtheorem{theorem}{Thm}
\newtheorem{corollary}{Col}
\newtheorem{lemma}{Lem}
\newtheorem{definition}{Def}
\newtheorem{proposition}{Prop}

\tcbuselibrary{most}

\renewcommand{\labelitemi}{$\circ$}
\renewcommand{\labelitemii}{$\triangleright$}

\ProvideDocumentCommand\floor{m}{\lfloor {#1} \rfloor}
\ProvideDocumentCommand{\where}{}{\mathrel{}\middle|\mathrel{}}
\ProvideDocumentCommand{\when}{m}{\quad({#1})}
\ProvideDocumentCommand{\adjoint}{}{\mathbf{\ast}}
\ProvideDocumentCommand{\conjugation}{}{\mathbf{\ast}}
\NewDocumentCommand{\Jacobi}{m m}{\frac{\partial ({#1})}{\partial ({#2})}}
\NewDocumentCommand{\JacobiPolar}{O{x,y}}{\frac{\partial ({#1})}{\partial (r,\theta)}}
\DeclareMathOperator{\Ker}{Ker}
\DeclareMathOperator{\Img}{Im}
\DeclareMathOperator{\res}{Res}
\DeclareMathOperator{\Dom}{Dom}
\DeclareMathOperator{\Ran}{Ran}

\DeclareMathOperator{\exd}{d}

\DeclareFontShape{JY2}{mc}{m}{it}{<->ssub*mc/m/n}{}
\DeclareFontShape{JY2}{mc}{m}{sl}{<->ssub*mc/m/n}{}
\DeclareFontShape{JY2}{mc}{m}{sc}{<->ssub*mc/m/n}{}
\DeclareFontShape{JY2}{gt}{m}{it}{<->ssub*gt/m/n}{}
\DeclareFontShape{JY2}{gt}{m}{sl}{<->ssub*gt/m/n}{}
\DeclareFontShape{JY2}{mc}{b}{it}{<->ssub*mc/bx/it}{}
\DeclareFontShape{JY2}{mc}{bx}{it}{<->ssub*gt/m/n}{}
\DeclareFontShape{JY2}{mc}{bx}{sl}{<->ssub*gt/m/n}{}
\DeclareFontShape{JT2}{mc}{m}{it}{<->ssub*mc/m/n}{}
\DeclareFontShape{JT2}{mc}{m}{sl}{<->ssub*mc/m/n}{}
\DeclareFontShape{JT2}{mc}{m}{sc}{<->ssub*mc/m/n}{}
\DeclareFontShape{JT2}{gt}{m}{it}{<->ssub*gt/m/n}{}
\DeclareFontShape{JT2}{gt}{m}{sl}{<->ssub*gt/m/n}{}
\DeclareFontShape{JT2}{mc}{b}{it}{<->ssub*gt/b/n}{}
\DeclareFontShape{JT2}{mc}{bx}{it}{<->ssub*gt/m/n}{}
\DeclareFontShape{JT2}{mc}{bx}{sl}{<->ssub*gt/m/n}{}

\DeclareFontShape{J30}{mc}{b}{it}{<->ssub*mc/bx/it}{}
\DeclareFontShape{J30}{mc}{b}{n}{<->ssub*mc/bx/n}{}
\DeclareFontShape{J20}{mc}{b}{it}{<->ssub*mc/bx/it}{}
\DeclareFontShape{J20}{mc}{b}{n}{<->ssub*mc/bx/n}{}