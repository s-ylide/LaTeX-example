%& -shell-escape -enable-write18 
\documentclass[uplatex,dvipdfmx]{jsarticle}
\usepackage[utf8]{inputenc}

\usepackage[dvipdfmx]{graphicx}
\usepackage[dvipdfmx]{color}
\usepackage[driver=dvipdfm,hmargin=19.05truemm,vmargin=25.40truemm]{geometry}
\usepackage{colortbl}
\usepackage{tcolorbox}
\usepackage{varwidth}
\usepackage{xcolor}
\PassOptionsToPackage{dvipsnames}{xcolor}

\usepackage{amsmath,amsfonts,amssymb,amsthm}
\usepackage{bm}
\usepackage[italicdiff]{physics}
\usepackage{mathtools}
\mathtoolsset{showonlyrefs=true}
\usepackage[unicode, dvipdfmx]{hyperref}
\usepackage{pxjahyper}
\hypersetup{
setpagesize=false,
    bookmarksnumbered=true,
    bookmarksopen=true,
    colorlinks=true,
    linkcolor=cyan,
    citecolor=red,
}
\usepackage{mathrsfs}
\usepackage{bussproofs}
\usepackage{enumerate}
\usepackage{pxrubrica}
\usepackage{tipa}
\usepackage{ascmac}
\usepackage{caption}
\usepackage[subrefformat=parens]{subcaption}
\usepackage{listings,jvlisting}
\usepackage{tikz}
\usetikzlibrary{math,patterns,intersections,calc,arrows,graphs}
\usepackage{float}
\usepackage{xparse}
\usepackage{url}
\usepackage{fancyhdr}
\usepackage{multicol}
\usepackage{siunitx}
\usepackage{hhline}
\usepackage{bxbase}
\usepackage[mark=***]{sectionbreak}
\usepackage[geometry]{ifsym}
\usepackage[prefernoncjk]{pxcjkcat}
\usepackage[LGR,T2A,T3,T1]{fontenc}
\usepackage[greek,latin,english,russian,japanese]{pxbabel}

\allowdisplaybreaks


\theoremstyle{definition}
\newtheorem{theorem}{Thm}
\newtheorem{corollary}{Col}
\newtheorem{lemma}{Lem}
\newtheorem{definition}{Def}
\newtheorem{proposition}{Prop}

\tcbuselibrary{most}

\renewcommand{\labelitemi}{$\circ$}
\renewcommand{\labelitemii}{$\triangleright$}

\ProvideDocumentCommand\floor{m}{\lfloor {#1} \rfloor}
\ProvideDocumentCommand{\where}{}{\mathrel{}\middle|\mathrel{}}
\ProvideDocumentCommand{\when}{m}{\quad({#1})}
\ProvideDocumentCommand{\adjoint}{}{\mathbf{\ast}}
\ProvideDocumentCommand{\conjugation}{}{\mathbf{\ast}}
\NewDocumentCommand{\Jacobi}{m m}{\frac{\partial ({#1})}{\partial ({#2})}}
\NewDocumentCommand{\JacobiPolar}{O{x,y}}{\frac{\partial ({#1})}{\partial (r,\theta)}}
\DeclareMathOperator{\Ker}{Ker}
\DeclareMathOperator{\Img}{Im}
\DeclareMathOperator{\res}{Res}
\DeclareMathOperator{\Dom}{Dom}
\DeclareMathOperator{\Ran}{Ran}

\DeclareMathOperator{\exd}{d}

\DeclareFontShape{JY2}{mc}{m}{it}{<->ssub*mc/m/n}{}
\DeclareFontShape{JY2}{mc}{m}{sl}{<->ssub*mc/m/n}{}
\DeclareFontShape{JY2}{mc}{m}{sc}{<->ssub*mc/m/n}{}
\DeclareFontShape{JY2}{gt}{m}{it}{<->ssub*gt/m/n}{}
\DeclareFontShape{JY2}{gt}{m}{sl}{<->ssub*gt/m/n}{}
\DeclareFontShape{JY2}{mc}{b}{it}{<->ssub*mc/bx/it}{}
\DeclareFontShape{JY2}{mc}{bx}{it}{<->ssub*gt/m/n}{}
\DeclareFontShape{JY2}{mc}{bx}{sl}{<->ssub*gt/m/n}{}
\DeclareFontShape{JT2}{mc}{m}{it}{<->ssub*mc/m/n}{}
\DeclareFontShape{JT2}{mc}{m}{sl}{<->ssub*mc/m/n}{}
\DeclareFontShape{JT2}{mc}{m}{sc}{<->ssub*mc/m/n}{}
\DeclareFontShape{JT2}{gt}{m}{it}{<->ssub*gt/m/n}{}
\DeclareFontShape{JT2}{gt}{m}{sl}{<->ssub*gt/m/n}{}
\DeclareFontShape{JT2}{mc}{b}{it}{<->ssub*gt/b/n}{}
\DeclareFontShape{JT2}{mc}{bx}{it}{<->ssub*gt/m/n}{}
\DeclareFontShape{JT2}{mc}{bx}{sl}{<->ssub*gt/m/n}{}

\DeclareFontShape{J30}{mc}{b}{it}{<->ssub*mc/bx/it}{}
\DeclareFontShape{J30}{mc}{b}{n}{<->ssub*mc/bx/n}{}
\DeclareFontShape{J20}{mc}{b}{it}{<->ssub*mc/bx/it}{}
\DeclareFontShape{J20}{mc}{b}{n}{<->ssub*mc/bx/n}{}
\ProvideDocumentCommand{\titleof}{m}{

\title{ここに講義タイトルを入力 \\ {#1}レポート課題}
\author{ここに名前を入力}
\date{}
}
\titleof{11/09}

\begin{document}
    
\maketitle

\begin{enumerate}[(1)]
    \item 
    \begin{enumerate}[(i)]
        \item $\sqrt{x^2-1}=t-x$とおくと,$t>0$で
        \begin{align}
            \sqrt{x^2-1}&=t-x\\
            x^2-1&=t^2-2tx+x^2\\
            x&=\frac{t^2+1}{2t}
        \end{align}
        これを$t$で微分すると,
        \begin{align}
            \dv{x}{t} &=\frac{1}{2}(1-\frac{1}{t^2})\\
            &=\frac{t^2-1}{2t^2}\\
            &=\frac{2x^2-2+2x\sqrt{x^2-1}}{2t^2}\\
            &=\frac{\sqrt{x^2-1}(\sqrt{x^2-1}+x)}{t^2}\\
            &=\frac{(t-x)t}{t^2}\\
            &=\frac{t-x}{t}
        \end{align}
        である.よって,
        \begin{align}
            \int \frac{\dd{x}}{\sqrt{x^2-1}}
            &= \int \frac{\dv{x}{t}}{t-x}\dd{t}\\
            &= \int \frac{1}{t}\dd{t}\\
            &= \log{t} + C \qq{($C$は任意)}\\
            &= \log(x+\sqrt{x^2-1}) + C
        \end{align}
        である.
        \item 
        \begin{align}
            I
            &=\int^2_0 \frac{\dd{x}}{\sqrt{\abs{x^2-1}}}\\
            &=\int^1_0 \frac{\dd{x}}{\sqrt{1-x^2}} + \int^2_1 \frac{\dd{x}}{\sqrt{x^2-1}}\\
            &=\lim_{s\to 1-0}\eval[\arcsin{x}|^s_{x\coloneqq 0} + \lim_{s\to 1+0}\eval[\log(x+\sqrt{x^2-1}) |^2_{x\coloneqq s}\\
            &=\frac{\pi}{2} + \log(2+\sqrt{3})
        \end{align}
        である.
    \end{enumerate}
    \item 
    \begin{enumerate}[(i)]
        \item $\displaystyle \lim_{x\to\infty}f(x)=0$を示す.
        \begin{proof}
            \begin{equation}
                \exists \varepsilon _{> 0}\, \exists x\, [f(x) = -\varepsilon]
            \end{equation}
            と仮定する.

            $\varepsilon > 0$をとる.
            $f(x)= -\varepsilon$となる$x$を1つとり,$s$とする.$f(x)$は広義単調減少ゆえ
            \begin{equation}
                t \ge s \to f(t)\le f(s) = -\varepsilon
            \end{equation}
            であるから,$t>s$に対し
            \begin{align}
                \int^t_0 f(x) \dd{x} 
                &= \int^s_0 f(x) \dd{x} + \int^t_s f(x) \dd{x} \\ 
                &\le \int^s_0 f(x) \dd{x} + \int^t_s (-\varepsilon)\dd{x} \\ 
                &= \int^s_0 f(x) \dd{x} -(t-s)\varepsilon \\ 
                &= -\infty\quad (\qq*{as} t\to \infty)
            \end{align}
            となるが,これは$I$の収束に反するので矛盾である.よって
            \begin{align}
                \forall \varepsilon _{> 0}\, &\forall x\, [f(x) \ne -\varepsilon]\\
                \therefore &\forall x\, [f(x) \ge 0] \label{eq:fx_ge_0}
            \end{align}
            である.すると,$f(x)$の値域の下限が存在するので,それを$l$とおくと明らかに$l\ge 0$.
            $t>0$に対し
            \begin{align}
                \int^t_0 f(x) \dd{x} 
                &\ge \int^t_0 l \dd{x} \\
                &= lt
            \end{align}
            であって右辺は$t\to\infty$の極限で収束するから$l=0$である.

            改めて$\varepsilon > 0$をとる.
            下限の定義より,$\varepsilon$に対して$\delta\ge 0$が存在して$f(\delta)<l+\varepsilon=\varepsilon$を満たす.

            $f(x)$は広義単調減少ゆえ
            \begin{equation}
                x > \delta \to x \ge \delta \to f(x)\le f(\delta) < \varepsilon
            \end{equation}
            であるから,\eqref{eq:fx_ge_0}式と合わせて$\abs{f(x)}<\varepsilon$,すなわち
            \begin{equation}
                \lim_{x\to\infty}f(x)=0
            \end{equation}
            である.
        \end{proof}
        \item 
        \begin{equation}
            \lim_{x\to\infty}f(x)=0
        \end{equation}
        すなわち
        \begin{equation}
            \forall \varepsilon _{>0} \, \exists \delta [x>\delta \to \abs{f(x)}<\varepsilon]
        \end{equation}
        であり,\eqref{eq:fx_ge_0}式と合わせて$\varepsilon \coloneqq 1$に対して$\delta $が存在して
        \begin{align}
            x>\delta &\to f(x)<1\\
            \therefore x>\delta &\to \{f(x)\}^2 < f(x)<1
        \end{align}
        を満たす.したがって
        \begin{align}
            0\le \int^\infty_\delta \{f(x)\}^2 \dd{x} 
            &< \int^\infty_\delta f(x) \dd{x} \\
            &\le \int^\infty_0 f(x) \dd{x} \\
            &= I 
        \end{align}
        であるので,$\displaystyle J=\int^\infty_0 \{f(x)\}^2 \dd{x}$の上界として$\displaystyle I+\int^\delta_0 f(x)\dd{x}$がある.

        関数$\displaystyle j(t)=\int^t_0 \{f(x)\}^2 \dd{x} $を考える.
        \begin{itemize}
            \item $\exists x \ [f(x)=0]$のとき\\
            そのような$x$をとり$c$とする.$J=j(c)$より収束する.

            \item そうでないとき,すなわち$f(x)>0$のとき\\
            $j(t)$は単調増加である.また,$j(t)$の値域は上に有界なので$\displaystyle J=\lim_{t\to\infty}j(t)$は収束する.
        \end{itemize}
        以上から$\displaystyle J=\int^\infty_0 \{f(x)\}^2 \dd{x}$は収束する.
    \end{enumerate}
    \item \
    \begin{itemize}
        \item $\displaystyle I=\int^1_0 x^5 \qty(\log \frac{1}{x^4})^\frac{3}{2} \dd{x} = \frac{\sqrt{\pi}}{6\sqrt{6}}$
        
        $\displaystyle x=e^{-\frac{1}{6}t}$とおく.
        \begin{align}
            I
            &=\int^1_0 x^5 \qty(\log \frac{1}{x^4})^\frac{3}{2} \dd{x}\\
            &=\int^0_{\infty} e^{-\frac{5}{6}t} \qty(\log e^{\frac{2}{3}t})^\frac{3}{2} \qty(-\frac{1}{6})e^{-\frac{1}{6}t}\dd{t}\\
            &=\frac{1}{6}\int^\infty_0 e^{-t} \qty(\frac{2}{3}t)^\frac{3}{2}\dd{t}\\
            &=\frac{1}{6}\qty(\frac{2}{3})^\frac{3}{2}\int^\infty_0 e^{-t} t^\frac{3}{2}\dd{t}\\
            &=\frac{2}{9\sqrt{6}}\Gamma(\frac{5}{2})\\
            &=\frac{2}{9\sqrt{6}} \cdot \frac{3}{2} \Gamma(\frac{3}{2})\\
            &=\frac{2}{9\sqrt{6}} \cdot \frac{3}{2} \cdot \frac{1}{2} \Gamma(\frac{1}{2})\\
            &=\frac{\sqrt{\pi}}{6\sqrt{6}} \\
        \end{align}
        である.

        \item $J=\cfrac{\displaystyle \int^1_0 (1-x^3)^{-\frac{1}{2}} \dd{x}}{\displaystyle \int^1_0 (1-x^3)^{\frac{1}{2}} \dd{x}} = \dfrac{5}{3}$
        
        $1-x^3 = t$とおく.
        \begin{align}
            J
            &=\dfrac{ \int^1_0 (1-x^3)^{-\frac{1}{2}} \dd{x}}{ \int^1_0 (1-x^3)^{\frac{1}{2}} \dd{x}}\\
            &=\dfrac{ \int^0_1 t^{-\frac{1}{2}} \qty(-\frac{1}{3})(1-t)^{-\frac{2}{3}}\dd{t}}{ \int^0_1 t^{\frac{1}{2}}\qty(-\frac{1}{3})(1-t)^{-\frac{2}{3}} \dd{t}}\\
            &=\dfrac{ \int^1_0 t^{-\frac{1}{2}} (1-t)^{-\frac{2}{3}}\dd{t}}{ \int^1_0 t^{\frac{1}{2}} (1-t)^{-\frac{2}{3}} \dd{t}}\\
            &=\dfrac{B(\frac12,\frac13)}{B(\frac32,\frac13)}\\
            &=\dfrac{\Gamma(\frac12)\Gamma(\frac13)}{\Gamma(\frac56)}\cdot \dfrac{\Gamma(\frac{11}{6})}{\Gamma(\frac32)\Gamma(\frac13)}\\
            &=\dfrac{\Gamma(\frac12)}{\Gamma(\frac56)}\cdot \dfrac{\frac{5}{6}\Gamma(\frac{5}{6})}{\frac{1}{2}\Gamma(\frac12)}\\
            &=\frac53\\
        \end{align}
    \end{itemize}
    \item \begin{enumerate}[(a)]
        \item 
        \begin{align}
            \qq{$AB$の傾き} h_1 &\coloneqq \frac{f(b)-f(a)}{b-a}\\
            \qq{$BC$の傾き} h_2 &\coloneqq \frac{f(c)-f(b)}{c-b}\\
            \qq{$AC$の傾き} h_3 &\coloneqq \frac{f(c)-f(a)}{c-a}
        \end{align}
        とする.$f(x)$は下に凸なので
        \begin{align}
            \frac{f(a)(c-b)+f(c)(b-a)}{(c-b) + (b-a)} &\ge f(b)\\
            \frac{f(a)(c-b)+f(c)(b-a)}{c-a} - f(a) &\ge f(b) - f(a)\\
            f(c)\frac{b-a}{c-a} - f(a)\frac{b-a}{c-a} &\ge f(b) - f(a)\\
            \frac{f(c)-f(a)}{c-a} &\ge \frac{f(b)-f(a)}{b-a}\\
            \therefore h_3 &\ge h_1
        \end{align}
        である.
        $a,c$を入れ替えることで,$h_3\le h_2$を得る.よって
        \begin{equation}
            h_1 \le h_3 \le h_2
        \end{equation}
        がわかる.

        \begin{lemma}
            $\forall x_1<\forall x_2$に対し,
            $m=\dfrac{g(x_2)-g(x_1)}{x_2-x_1}$とすると
            \begin{equation}
                g^\prime(x_1)<m, g^\prime(x_2)>m
            \end{equation}
            である.
            \begin{proof}
                背理法で示す.

                $g^\prime(x_1)\ge m$と仮定する.
                $g^\prime(x)$は単調増加ゆえ
                \begin{align}
                    x>x_1 &\to g^\prime(x)>g^\prime(x_1)\ge m\\
                    \therefore \int^{x_2}_{x_1} g^\prime(x)\dd{x} 
                    &> \int^{x_2}_{x_1} g^\prime(x_1)\dd{x}\\
                    &= (x_2-x_1)g^\prime(x_1)\\
                    &\ge (x_2-x_1)m\\
                    &= g(x_2)-g(x_1)
                \end{align}
                であるが,これは$\displaystyle\int^{x_2}_{x_1} g^\prime(x)\dd{x} = g(x_2)-g(x_1)$に矛盾する.したがって$g^\prime(x_1)< m$である.

                同様にして$g^\prime(x_2)> m$が示される.
            \end{proof}
        \end{lemma}

        $g(x)\in C^2$は$g^{\prime\prime}(x)>0$とする.このとき,$\forall x_1<\forall x_2, \forall t \in (0,1)$に対し
        \begin{equation}
            (1-t)g(x_1)+tg(x_2)>g((1-t)x_1+tx_2) \qq{(狭義の凸)}
        \end{equation}
        が成り立つことを示す.
        \begin{proof}
            $x_1<x_2$を任意にとる.
            \begin{equation}
                h(t)=(1-t)g(x_1)+tg(x_2) - g((1-t)x_1+tx_2)
            \end{equation}
            とおく.
            \begin{equation}
                h^{\prime}(t)=g(x_2)-g(x_1)-\qty((x_2-x_1)g^{\prime}(x_1+t(x_2-x_1)))
            \end{equation}
            から$h^{\prime}(t)$は$[0,1]$で連続で,中間値の定理より,$c \in (0,1)$が存在して
            \begin{equation}
                h^{\prime}(c) = 0
            \end{equation}
            を満たすから,そのような$c$を一つとる.
            \begin{equation}
                h^{\prime\prime}(t)=-(x_2-x_1)^2\qty(g^{\prime\prime}(x_1+t(x_2-x_1)))<0
            \end{equation}
            から$h^{\prime}(t)$は単調減少なので,$h(0)=0, h(1)=0, h^{\prime}(0)>0, h^{\prime}(1)<0$とあわせて
            \begin{align}
                0<t<c &\to h(t)>0\\
                c<t<1 &\to h(t)>0
            \end{align}
            である.
            また,
            \begin{align}
                h(c)
                &=h(0)+\int^c_0 h^\prime(t)\dd{t}\\
                &=\int^c_0 h^\prime(t)\dd{t}
            \end{align}
            であるが$h^\prime(0)>0, 0\le t\le c \to h^\prime(t)\ge 0$により$h(c)>0$が従う.

            以上から
            \begin{equation}
                (1-t)g(x_1)+tg(x_2)>g((1-t)x_1+tx_2)
            \end{equation}
            が成り立つ.
        \end{proof}
        また,先の$h(t)$で$t\coloneqq 0, 1$とすると不等式に等号がついたもの,すなわち

        $\forall x_1<\forall x_2, \forall t \in [0,1]$に対して
        \begin{equation}
            (1-t)g(x_1)+tg(x_2)\ge g((1-t)x_1+tx_2)
        \end{equation}
        が得られる.よって$g(x)$のグラフは下に凸である.
    \end{enumerate}
\end{enumerate}

\end{document}
        