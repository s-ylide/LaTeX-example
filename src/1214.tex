%& -shell-escape -enable-write18 
\documentclass[uplatex,dvipdfmx]{jsarticle}
\usepackage[utf8]{inputenc}

\usepackage[dvipdfmx]{graphicx}
\usepackage[dvipdfmx]{color}
\usepackage[driver=dvipdfm,hmargin=19.05truemm,vmargin=25.40truemm]{geometry}
\usepackage{colortbl}
\usepackage{tcolorbox}
\usepackage{varwidth}
\usepackage{xcolor}
\PassOptionsToPackage{dvipsnames}{xcolor}

\usepackage{amsmath,amsfonts,amssymb,amsthm}
\usepackage{bm}
\usepackage[italicdiff]{physics}
\usepackage{mathtools}
\mathtoolsset{showonlyrefs=true}
\usepackage[unicode, dvipdfmx]{hyperref}
\usepackage{pxjahyper}
\hypersetup{
setpagesize=false,
    bookmarksnumbered=true,
    bookmarksopen=true,
    colorlinks=true,
    linkcolor=cyan,
    citecolor=red,
}
\usepackage{mathrsfs}
\usepackage{bussproofs}
\usepackage{enumerate}
\usepackage{pxrubrica}
\usepackage{tipa}
\usepackage{ascmac}
\usepackage{caption}
\usepackage[subrefformat=parens]{subcaption}
\usepackage{listings,jvlisting}
\usepackage{tikz}
\usetikzlibrary{math,patterns,intersections,calc,arrows,graphs}
\usepackage{float}
\usepackage{xparse}
\usepackage{url}
\usepackage{fancyhdr}
\usepackage{multicol}
\usepackage{siunitx}
\usepackage{hhline}
\usepackage{bxbase}
\usepackage[mark=***]{sectionbreak}
\usepackage[geometry]{ifsym}
\usepackage[prefernoncjk]{pxcjkcat}
\usepackage[LGR,T2A,T3,T1]{fontenc}
\usepackage[greek,latin,english,russian,japanese]{pxbabel}

\allowdisplaybreaks


\theoremstyle{definition}
\newtheorem{theorem}{Thm}
\newtheorem{corollary}{Col}
\newtheorem{lemma}{Lem}
\newtheorem{definition}{Def}
\newtheorem{proposition}{Prop}

\tcbuselibrary{most}

\renewcommand{\labelitemi}{$\circ$}
\renewcommand{\labelitemii}{$\triangleright$}

\ProvideDocumentCommand\floor{m}{\lfloor {#1} \rfloor}
\ProvideDocumentCommand{\where}{}{\mathrel{}\middle|\mathrel{}}
\ProvideDocumentCommand{\when}{m}{\quad({#1})}
\ProvideDocumentCommand{\adjoint}{}{\mathbf{\ast}}
\ProvideDocumentCommand{\conjugation}{}{\mathbf{\ast}}
\NewDocumentCommand{\Jacobi}{m m}{\frac{\partial ({#1})}{\partial ({#2})}}
\NewDocumentCommand{\JacobiPolar}{O{x,y}}{\frac{\partial ({#1})}{\partial (r,\theta)}}
\DeclareMathOperator{\Ker}{Ker}
\DeclareMathOperator{\Img}{Im}
\DeclareMathOperator{\res}{Res}
\DeclareMathOperator{\Dom}{Dom}
\DeclareMathOperator{\Ran}{Ran}

\DeclareMathOperator{\exd}{d}

\DeclareFontShape{JY2}{mc}{m}{it}{<->ssub*mc/m/n}{}
\DeclareFontShape{JY2}{mc}{m}{sl}{<->ssub*mc/m/n}{}
\DeclareFontShape{JY2}{mc}{m}{sc}{<->ssub*mc/m/n}{}
\DeclareFontShape{JY2}{gt}{m}{it}{<->ssub*gt/m/n}{}
\DeclareFontShape{JY2}{gt}{m}{sl}{<->ssub*gt/m/n}{}
\DeclareFontShape{JY2}{mc}{b}{it}{<->ssub*mc/bx/it}{}
\DeclareFontShape{JY2}{mc}{bx}{it}{<->ssub*gt/m/n}{}
\DeclareFontShape{JY2}{mc}{bx}{sl}{<->ssub*gt/m/n}{}
\DeclareFontShape{JT2}{mc}{m}{it}{<->ssub*mc/m/n}{}
\DeclareFontShape{JT2}{mc}{m}{sl}{<->ssub*mc/m/n}{}
\DeclareFontShape{JT2}{mc}{m}{sc}{<->ssub*mc/m/n}{}
\DeclareFontShape{JT2}{gt}{m}{it}{<->ssub*gt/m/n}{}
\DeclareFontShape{JT2}{gt}{m}{sl}{<->ssub*gt/m/n}{}
\DeclareFontShape{JT2}{mc}{b}{it}{<->ssub*gt/b/n}{}
\DeclareFontShape{JT2}{mc}{bx}{it}{<->ssub*gt/m/n}{}
\DeclareFontShape{JT2}{mc}{bx}{sl}{<->ssub*gt/m/n}{}

\DeclareFontShape{J30}{mc}{b}{it}{<->ssub*mc/bx/it}{}
\DeclareFontShape{J30}{mc}{b}{n}{<->ssub*mc/bx/n}{}
\DeclareFontShape{J20}{mc}{b}{it}{<->ssub*mc/bx/it}{}
\DeclareFontShape{J20}{mc}{b}{n}{<->ssub*mc/bx/n}{}
\ProvideDocumentCommand{\titleof}{m}{

\title{ここに講義タイトルを入力 \\ {#1}レポート課題}
\author{ここに名前を入力}
\date{}
}

\titleof{12/14}

\begin{document}

\maketitle

\begin{enumerate}[(1)]
    \item \begin{enumerate}[(i)]
        \item \begin{align}
            E &= [0, \infty) \times [0, 2\pi)\\
            \Phi &: (r,\theta) \mapsto (x,y)=(r\cos\theta, r\sin\theta)
        \end{align}
        とすると,$\Phi(E)=\mathbb{R}^2$であり
        \begin{equation}
            \Jacobi{x,y}{r,\theta}=r
        \end{equation}
        は$E\setminus (\qty{0}\times [0, 2\pi))$で$\displaystyle \Jacobi{x,y}{r,\theta}\ne 0$であるから全単射である.また$m(\qty{0}\times [0, 2\pi))=0$.よって
        \begin{equation}
            \Phi_{\restriction E} : E \to \mathbb{R}^2
        \end{equation}
        はほぼ全単射である.したがって
        \begin{align}
            \iint_{\mathbb{R}^2}\frac{1}{(x^2+y^2+1)^s}\dd{x}\dd{y}
            &=\iint_E \frac{1}{(r^2+1)^s}\Jacobi{x,y}{r,\theta} \dd{r}\dd{\theta}\\
            &=\int_0^{2\pi}\dd{\theta} \int_0^\infty \frac{1}{(r^2+1)^s}r \dd{r}\\
            &=2\pi \int_0^\infty \frac{1}{(r^2+1)^s}r \dd{r}\\
            &=\begin{dcases}
                \pi \eval[\log(r^2+1)|_{r\coloneqq 0}^\infty  &(s=1\text{のとき})\\
                \frac{\pi}{-(s-1)}  \eval[ (r^2+1)^{-(s-1)} |_{r\coloneqq 0}^\infty &(s\ne 1\text{のとき})
            \end{dcases}\\
            &=\begin{dcases}
                \infty \qq{(発散)} &(s=1\text{のとき})\\
                \frac{\pi}{-(s-1)} \qty(\lim_{r\to\infty} (r^2+1)^{-(s-1)} - 1) &(s\ne 1\text{のとき})
            \end{dcases}\\
            &=\begin{dcases}
                \infty \qq{(発散)} &(0 < s \le 1\text{のとき})\\
                \frac{\pi}{(s-1)} \qq{(収束)} &(s > 1\text{のとき})
            \end{dcases}
        \end{align}
        であるから,$\displaystyle \iint_{\mathbb{R}^2}\frac{1}{(x^2+y^2+1)^s}\dd{x}\dd{y}$が収束する条件は
        \begin{equation}
            s > 1
        \end{equation}
        であって,その値は
        \begin{equation}
            \dfrac{\pi}{(s-1)}
        \end{equation}
        である.
        \item \begin{equation}
            D_r = \qty{(x, y)\where 0\le x^2+y^2 \le r^2}
        \end{equation}
        とする.
        $(x,y)\ne (0,0)$のとき
        \begin{equation}
            \frac{1}{(x^2+y^2)^s}>0
        \end{equation}
        であるから,
        \begin{itemize}
            \item $0<s< 1$のとき
            \begin{align}
                \iint_{\abs{x}\le 1, \abs{y}\le 1} \frac{1}{(x^2+y^2)^s} \dd{x}\dd{y} 
                &< \iint_{D_{\sqrt{2}}} \frac{1}{(x^2+y^2)^s} \dd{x}\dd{y} \\
                &= \iint_{\substack{0\le r \le \sqrt{2}\\0\le\theta<2\pi}} \frac{1}{(r^2)^s} \Jacobi{x,y}{r,\theta} \dd{r}\dd{\theta}\\
                &= \iint_{\substack{0\le r \le \sqrt{2}\\0\le\theta<2\pi}} r^{1-2s} \dd{r}\dd{\theta}\\
                &= \int_0^{2\pi}\dd{\theta} \int_0^{\sqrt{2}} r^{1-2s} \dd{r}\\
                &= 2\pi \eval[\dfrac{1}{2(1-s)}r^{2(1-s)}|_{r\coloneqq 0}^{\sqrt{2}} \\
                &= \dfrac{\pi}{1-s} \qty(2^{(1-s)})\\
                &= \dfrac{2^{(1-s)}\pi}{1-s} \qq{(収束)} 
            \end{align}
            \item $s\ge 1$のとき
            \begin{align}
                \iint_{\abs{x}\le 1, \abs{y}\le 1} \frac{1}{(x^2+y^2)^s} \dd{x}\dd{y} 
                &\ge \iint_{D_1} \frac{1}{(x^2+y^2)^s} \dd{x}\dd{y} \\
                &= \iint_{\substack{0\le r \le 1\\0\le\theta<2\pi}} \frac{1}{(r^2)^s} \Jacobi{x,y}{r,\theta} \dd{r}\dd{\theta}\\
                &= \iint_{\substack{0\le r \le 1\\0\le\theta<2\pi}} r^{1-2s} \dd{r}\dd{\theta}\\
                &= \int_0^{2\pi}\dd{\theta} \int_0^1 r^{1-2s} \dd{r}\\
                &= \begin{dcases}
                    2\pi \eval[\log r|_{r\coloneqq +0}^1 &(s=1\text{のとき})\\
                    2\pi \eval[\dfrac{1}{2(1-s)}r^{2(1-s)}|_{r\coloneqq +0}^1 &(s> 1\text{のとき})\\
                \end{dcases}\\
                &= \infty \qq{(発散)}
            \end{align}
        \end{itemize}
        であるので,$\displaystyle \iint_{\abs{x}\le 1, \abs{y}\le 1} \frac{1}{(x^2+y^2)^s} \dd{x}\dd{y}$が収束する条件は
        \begin{equation}
            (0<)s< 1
        \end{equation}
        である.
    \end{enumerate}
    \item $F(t)$の増減を調べる.
    \begin{align}
        \pdv{t} F(t)
        &=\int_1^2 \pdv{t}(\frac{\sin tx}{x})\dd{x}\\
        &=\int_1^2 \frac{1}{x} x\cos tx \dd{x}\\
        &=\int_1^2 \cos tx \dd{x}\\
        &=\begin{dcases}
            \int_1^2 1 \dd{x} &(t=0\text{のとき})\\
            \frac{1}{t}\eval[\sin tx|_{x\coloneqq 1}^2&(t\ne 0\text{のとき})
        \end{dcases}\\
        &=\begin{dcases}
            1 &(t=0\text{のとき})\\
            \frac{1}{t}(\sin 2t-\sin t)&(t\ne 0\text{のとき})
        \end{dcases}\\
        &=\begin{dcases}
            1 &(t=0\text{のとき})\\
            \frac{1}{t}\sin t(2\cos t- 1)&(t\ne 0\text{のとき})
        \end{dcases}
    \end{align}
    であるので,$F(t)$の増減は次の表のようになる.
    \begin{table}[H]
        \centering
        \begin{tabular}{|c|c|c|c|c|c|c|c|}
            \hline
            $t$ & 
            $(-\pi)$ &
            &
            $-\frac{\pi}{3}$ &
            &
            $\frac{\pi}{3}$ &
            &
            $(\pi)$
			\\
            \hhline{|=|=|=|=|=|=|=|=|}
			$\pdv{t} F(t)$ & & $-$ & $0$ & $+$ & $0$ & $-$ & \\ \hline
			$F(t)$ & & $\searrow $ & (極小) & $\nearrow $ & (極大) & $\searrow $ & \\ \hline
        \end{tabular}
        \caption{$F(t)$の増減表}
    \end{table}
    したがって
    \begin{equation}
        \begin{dcases}
            \text{極大を与える$t$の値}:t=\frac{\pi}{3}\\
            \text{極小を与える$t$の値}:t=-\frac{\pi}{3}
        \end{dcases}
    \end{equation}
    である.
    \item \begin{enumerate}[(a)]
        \item $\vb*x = \mqty(x\\y)$とし,$(\cdot, \cdot)$を$\mathbb{R}^2$の標準内積とする.このとき
        \begin{equation}
            f(\vb*x)\coloneqq ax^2+2bxy+cy^2=(\vb*x, A\vb*x)
        \end{equation}
        と書け,直交行列$P$が存在して
        \begin{align}
            \mqty(x\\y)&=P\mqty(y_1\\y_2)\\
            f(\vb*x)&=\lambda_1y_1^2+ \lambda_2y_2^2
        \end{align}
        を満たす.ただし,$\lambda_1, \lambda_2$は$A$の固有値である.このような$P$を1つとる.
        
        $P$を表現行列とする線形写像$g_P:\mathbb{R}^2\to \mathbb{R}^2$は全単射で, $\displaystyle \abs{\Jacobi{x,y}{y_1,y_2}}=\abs{P}\eqqcolon c \quad(\ne 0)$である.

        これらを用いて$I$を書き直すと
        \begin{align}
            I
            &=\iint_{\mathbb{R}^2} e^{f(\vb*x)}\dd{x}\dd{y}\\
            &=\iint_{\mathbb{R}^2} e^{\lambda_1y_1^2+ \lambda_2y_2^2}\abs{\Jacobi{x,y}{y_1,y_2}}\dd{y_1}\dd{y_2}\\
            &=c\int_{-\infty}^\infty e^{\lambda_1y_1^2}\dd{y_1} \int_{-\infty}^\infty e^{\lambda_2y_2^2} \dd{y_2}
        \end{align}
        ここで,次の広義積分の収束・発散は
        \begin{align}
            J
            &=\int_{-\infty}^\infty e^{\lambda x^2}\dd{x}\\
            &=\begin{dcases}
                \sqrt{\frac{\pi}{\lambda}} \qq{(収束)} &(\lambda<0\text{のとき})\\
                \infty \qq{(発散)} &(\lambda\ge 0\text{のとき})
            \end{dcases}
        \end{align}
        であったから,$I$が収束する条件は
        \begin{equation}
            \lambda_1 < 0 \land \lambda_2 < 0
        \end{equation}
        である.

        これを$a,b,c$で表すことを考える.
        $A$の固有方程式は
        \begin{align}
            \varphi_A(t)
            &=\mqty|a-t&b\\b&c-t|\\
            &=(a-t)(c-t)-b^2\\
            &=t^2-(a+c)t + ac-b^2
        \end{align}
        であって,$\lambda_1, \lambda_2$はこの根である.よって,
        \begin{align}
            &\lambda_1 < 0 \land \lambda_2 < 0\\
            \Longleftrightarrow\quad & \lambda_1 + \lambda_2 < 0 \land \lambda_1\lambda_2 > 0\\
            \Longleftrightarrow\quad & a + c < 0 \land ac-b^2 > 0
        \end{align}
        が$I$が収束するための必要十分条件である.
    \end{enumerate}
\end{enumerate}

\end{document}