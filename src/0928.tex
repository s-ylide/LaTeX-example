%& -shell-escape -enable-write18 
\documentclass[uplatex,dvipdfmx]{jsarticle}
\usepackage[utf8]{inputenc}

\usepackage[dvipdfmx]{graphicx}
\usepackage[dvipdfmx]{color}
\usepackage[driver=dvipdfm,hmargin=19.05truemm,vmargin=25.40truemm]{geometry}
\usepackage{colortbl}
\usepackage{tcolorbox}
\usepackage{varwidth}
\usepackage{xcolor}
\PassOptionsToPackage{dvipsnames}{xcolor}

\usepackage{amsmath,amsfonts,amssymb,amsthm}
\usepackage{bm}
\usepackage[italicdiff]{physics}
\usepackage{mathtools}
\mathtoolsset{showonlyrefs=true}
\usepackage[unicode, dvipdfmx]{hyperref}
\usepackage{pxjahyper}
\hypersetup{
setpagesize=false,
    bookmarksnumbered=true,
    bookmarksopen=true,
    colorlinks=true,
    linkcolor=cyan,
    citecolor=red,
}
\usepackage{mathrsfs}
\usepackage{bussproofs}
\usepackage{enumerate}
\usepackage{pxrubrica}
\usepackage{tipa}
\usepackage{ascmac}
\usepackage{caption}
\usepackage[subrefformat=parens]{subcaption}
\usepackage{listings,jvlisting}
\usepackage{tikz}
\usetikzlibrary{math,patterns,intersections,calc,arrows,graphs}
\usepackage{float}
\usepackage{xparse}
\usepackage{url}
\usepackage{fancyhdr}
\usepackage{multicol}
\usepackage{siunitx}
\usepackage{hhline}
\usepackage{bxbase}
\usepackage[mark=***]{sectionbreak}
\usepackage[geometry]{ifsym}
\usepackage[prefernoncjk]{pxcjkcat}
\usepackage[LGR,T2A,T3,T1]{fontenc}
\usepackage[greek,latin,english,russian,japanese]{pxbabel}

\allowdisplaybreaks


\theoremstyle{definition}
\newtheorem{theorem}{Thm}
\newtheorem{corollary}{Col}
\newtheorem{lemma}{Lem}
\newtheorem{definition}{Def}
\newtheorem{proposition}{Prop}

\tcbuselibrary{most}

\renewcommand{\labelitemi}{$\circ$}
\renewcommand{\labelitemii}{$\triangleright$}

\ProvideDocumentCommand\floor{m}{\lfloor {#1} \rfloor}
\ProvideDocumentCommand{\where}{}{\mathrel{}\middle|\mathrel{}}
\ProvideDocumentCommand{\when}{m}{\quad({#1})}
\ProvideDocumentCommand{\adjoint}{}{\mathbf{\ast}}
\ProvideDocumentCommand{\conjugation}{}{\mathbf{\ast}}
\NewDocumentCommand{\Jacobi}{m m}{\frac{\partial ({#1})}{\partial ({#2})}}
\NewDocumentCommand{\JacobiPolar}{O{x,y}}{\frac{\partial ({#1})}{\partial (r,\theta)}}
\DeclareMathOperator{\Ker}{Ker}
\DeclareMathOperator{\Img}{Im}
\DeclareMathOperator{\res}{Res}
\DeclareMathOperator{\Dom}{Dom}
\DeclareMathOperator{\Ran}{Ran}

\DeclareMathOperator{\exd}{d}

\DeclareFontShape{JY2}{mc}{m}{it}{<->ssub*mc/m/n}{}
\DeclareFontShape{JY2}{mc}{m}{sl}{<->ssub*mc/m/n}{}
\DeclareFontShape{JY2}{mc}{m}{sc}{<->ssub*mc/m/n}{}
\DeclareFontShape{JY2}{gt}{m}{it}{<->ssub*gt/m/n}{}
\DeclareFontShape{JY2}{gt}{m}{sl}{<->ssub*gt/m/n}{}
\DeclareFontShape{JY2}{mc}{b}{it}{<->ssub*mc/bx/it}{}
\DeclareFontShape{JY2}{mc}{bx}{it}{<->ssub*gt/m/n}{}
\DeclareFontShape{JY2}{mc}{bx}{sl}{<->ssub*gt/m/n}{}
\DeclareFontShape{JT2}{mc}{m}{it}{<->ssub*mc/m/n}{}
\DeclareFontShape{JT2}{mc}{m}{sl}{<->ssub*mc/m/n}{}
\DeclareFontShape{JT2}{mc}{m}{sc}{<->ssub*mc/m/n}{}
\DeclareFontShape{JT2}{gt}{m}{it}{<->ssub*gt/m/n}{}
\DeclareFontShape{JT2}{gt}{m}{sl}{<->ssub*gt/m/n}{}
\DeclareFontShape{JT2}{mc}{b}{it}{<->ssub*gt/b/n}{}
\DeclareFontShape{JT2}{mc}{bx}{it}{<->ssub*gt/m/n}{}
\DeclareFontShape{JT2}{mc}{bx}{sl}{<->ssub*gt/m/n}{}

\DeclareFontShape{J30}{mc}{b}{it}{<->ssub*mc/bx/it}{}
\DeclareFontShape{J30}{mc}{b}{n}{<->ssub*mc/bx/n}{}
\DeclareFontShape{J20}{mc}{b}{it}{<->ssub*mc/bx/it}{}
\DeclareFontShape{J20}{mc}{b}{n}{<->ssub*mc/bx/n}{}
\ProvideDocumentCommand{\titleof}{m}{

\title{ここに講義タイトルを入力 \\ {#1}レポート課題}
\author{ここに名前を入力}
\date{}
}
\titleof{10/05}

\begin{document}

\maketitle

\begin{enumerate}[(1)]
    \item d'Alembert の判定法を使う.
    \begin{itemize}
        \item $f(x)=\displaystyle\sum_{n=0}^\infty \dfrac{x^n}{\sqrt{n+1}}$について\\
        $a_n=\dfrac{1}{\sqrt{n+1}}$とする.
        \begin{align}
            r=\lim_{n\to\infty}\left|\dfrac{a_n}{a_{n+1}}\right|
            &=\lim_{n\to\infty}\dfrac{\sqrt{n+2}}{\sqrt{n+1}}\\
            &=\lim_{n\to\infty}\sqrt{\dfrac{1+\frac{2}{n}}{1+\frac{1}{n}}}\\
            &=1\\
            \therefore r &= 1
        \end{align}
        である.
        \item $g(x)=\displaystyle\sum_{n=0}^\infty{}_{2n}C_n x^n$について\\
        $b_n={}_{2n}C_n$とする.
        \begin{align}
            s=\lim_{n\to\infty}\left|\dfrac{b_n}{b_{n+1}}\right|
            &=\lim_{n\to\infty}\dfrac{(2n)!}{(n!)^2}\dfrac{(n+1)!^2}{(2n+2)!}\\
            &=\lim_{n\to\infty}\dfrac{(n+1)^2}{(2n+1)(2n+2)}\\
            &=\lim_{n\to\infty}\dfrac{(1+\frac{1}{n})^2}{(2+\frac{1}{n})(2+\frac{2}{n})}\\
            &=\dfrac{1}{4}\\
            \therefore s&=\dfrac{1}{4}
        \end{align}
        である.
    \end{itemize}
    \item 
    \begin{itemize}
        \item $f(r)$について\\
        $j(x)=\dfrac{1}{\sqrt{x+1}}$は$x>-1$で単調減少なので,
        簡単な面積比較から
        \begin{align}
            \sum_{n=0}^m j(n)>\int_0^{m}j(x)dx
            &=\eval[2\sqrt{x+1}|_0^m\\
            &=2\sqrt{m+1}-2
        \end{align}
        であり,$m\to\infty$とすると右辺は正の無限大に発散する.よって,
        \begin{align}
            \sum_{n=0}^\infty j(n)=\infty
        \end{align}
        となり,$f(r)$は正の無限大に発散する.
        \item $f(-r)$について\\
        $a_n=\dfrac{1}{\sqrt{n+1}}$は$n$の単調減少数列であり,
        $\displaystyle\lim_{n\to\infty}a_n=0$である.よって,9/28日の資料1 p.7の系2により,
        $f(-r)$は収束する.
        収束値は,Riemann zeta関数$\zeta(s)$を用いて$f(-r)=(1-\sqrt{2})\zeta\displaystyle\qty(\dfrac{1}{2})$と書ける.
        \item $g(s)$について\\
        $n\ge0$に対し,
        \begin{align}
            I_n := \int_0^\frac{\pi}{2}(\sin x)^ndx
        \end{align}
        と定義する.$n\ge 2$のとき
        \begin{align}
            I_n &= \int_0^\frac{\pi}{2}(\sin x)^{n-1}(\sin x)dx\\
            &=\eval[-(\sin x)^{n-1}\cos x|_0^\frac{\pi}{2}-\int_0^\frac{\pi}{2}-(n-1)(\sin x)^{n-2}(\cos x)^2dx\\
            &=(n-1)\qty(\int_0^\frac{\pi}{2}(\sin x)^{n-2}dx-\int_0^\frac{\pi}{2}(\sin x)^{n-2}(\sin x)^2dx)\\
            &=(n-1)(I_{n-2}+I_n)\\
            nI_n &= (n-1)I_{n-2}\label{eq:wallis_rec}\\
            nI_{n-1}I_n &= (n-1)I_{n-2}I_{n-1}\\
            &=1\cdot I_0I_1\\
            &=\frac{\pi}{2}
        \end{align}
        が成り立つ.また,\eqref{eq:wallis_rec}式から,
        \begin{align}
            I_{2n}
            &=\frac{2n-1}{2n}\cdot\frac{2n-3}{2n-2}\cdot\cdots\cdot\frac{1}{2}\cdot I_0\\
            &=\frac{(2n)(2n-1)}{(2n)^2}\cdot\frac{(2n-2)(2n-3)}{(2n-2)^2}\cdot\cdots\cdot\frac{2\cdot 1}{2^2}\cdot \frac{\pi}{2}\\
            &=\frac{(2n)!}{2^2n^2\cdot 2^2(n-1)^2\cdots 2^2}\cdot \frac{\pi}{2}\\
            &=\frac{(2n)!}{4^n(n!)^2}\cdot \frac{\pi}{2}\\
            &={}_{2n}C_n\qty(\frac{1}{4})^n\cdot \frac{\pi}{2}
        \end{align}
        である.ところで,$\displaystyle 0\le x\le \frac{\pi}{2}$に対し
        \begin{align}
            0\le \sin x\le 1
        \end{align}
        であるから,$n\ge 1$のとき
        \begin{equation}
            0\le(\sin x)^{n+1}\le (\sin x)^{n}\le(\sin x)^{n-1}
        \end{equation}
        \begin{equation}
            0\le I_{n+1}\le I_{n}\le I_{n-1}
        \end{equation}
        を得る.
        \begin{equation}
            I_{n}I_{n+1}\le (I_{n})^2\le I_{n-1}I_{n}
        \end{equation}
        \begin{equation}
            0< \frac{1}{n+1}\frac{\pi}{2}\le (I_{n})^2\le \frac{1}{n}\frac{\pi}{2}\label{eq:wallis_eval}
        \end{equation}
        $I_{n}>0$であるから
        \begin{equation}
            0< \frac{1}{\sqrt{n+1}}\sqrt{\frac{\pi}{2}}\le I_{n}
        \end{equation}
        $n\to2n$として
        \begin{equation}
            0< \frac{1}{\sqrt{2n+1}}\sqrt{\frac{\pi}{2}}\le I_{2n}={}_{2n}C_n\qty(\frac{1}{4})^n\cdot \frac{\pi}{2}
        \end{equation}
        \begin{equation}
            0< \frac{1}{\sqrt{2n+1}}\sqrt{\frac{2}{\pi}}\le I_{2n}={}_{2n}C_n\qty(\frac{1}{4})^n
        \end{equation}
        $n:=0,1,\dots,m$として両辺足し合わせて
        \begin{equation}
            0< \sqrt{\frac{2}{\pi}}\sum_{n=0}^{m} \frac{1}{\sqrt{2n+1}}\le \sum_{n=0}^{m} {}_{2n}C_n\qty(\frac{1}{4})^n
        \end{equation}
        簡単な面積比較から,
        \begin{equation}
            \int_0^m \frac{1}{\sqrt{2x+1}}dx = \sqrt{2m+1}-1 \le \sum_{n=0}^{m} \frac{1}{\sqrt{2n+1}}
        \end{equation}
        よって
        \begin{equation}
            \sqrt{\frac{2}{\pi}} (\sqrt{2m+1}-1) \le \sum_{n=0}^{m} {}_{2n}C_n\qty(\frac{1}{4})^n
        \end{equation}
        右辺は$m\to\infty$で正の無限大に発散するので,追い出しの原理により
        \begin{equation}
            \sum_{n=0}^{\infty} {}_{2n}C_n\qty(\frac{1}{4})^n=\infty \qq{(発散)}
        \end{equation}
        となる.$g(s)$は正の無限大に発散する.

        \item $g(-s)$について\\
        $f(-r)$のときと同様に,べき級数$g(-s)$の係数$b_n$が単調減少数列であることとそれが$0$に収束することから交代級数$g(-s)$が収束することを示す.
        まず,$\displaystyle\lim_{n\to\infty}{}_{2n}C_n\displaystyle\qty(\frac{1}{4})^n=0$であることを示す.\\
        先の\eqref{eq:wallis_eval}式の右辺は$n\to\infty$で$0$に収束するので,はさみうちの原理により
        \begin{align}
            \lim_{n\to\infty}I_{n}=0\label{eq:lim_of_wallis}
        \end{align}
        であるから,\eqref{eq:lim_of_wallis}と合わせて
        \begin{align}
            \lim_{n\to\infty}\frac{(2n)!}{4^n(n!)^2}\cdot \frac{\pi}{2}&=0\\
            \lim_{n\to\infty}\frac{(2n)!}{4^n(n!)^2}&=0\\
            \lim_{n\to\infty}{}_{2n}C_n\qty(\frac{1}{4})^n&=0
        \end{align}
        である.よって,$g(-s)$は収束する.
        
        べき級数
        \begin{equation}
            \frac{1}{\sqrt{{1-z^2}}}=\sum_{n=0}^{\infty}\frac{(2n-1)!!}{(2n)!!}z^{2n}=\sum_{n=0}^{\infty}{}_{2n}C_n\qty(\frac{1}{4})^n z^{2n}
        \end{equation}
        の収束半径が$1$であることと$g(-s)$が収束することから,$g(-s)$の値はこの級数に$z:=i$を代入することで得られ,
        \begin{align}
            \frac{1}{\sqrt{{1-i^2}}}
            &=\sum_{n=0}^{\infty}{}_{2n}C_n\qty(\frac{1}{4})^n i^{2n}\\
            \frac{1}{\sqrt{{1-(-1)}}}
            &=\sum_{n=0}^{\infty}{}_{2n}C_n\qty(\frac{1}{4})^n (-1)^{n}\\
            \frac{1}{\sqrt{2}}
            &=g(-s)
        \end{align}
        である.
    \end{itemize}
    \item 
    \begin{itemize}
        \item $h(1)$が発散すること\\
        $\sqrt{n}\ge\qty[\sqrt{n}]$であるから,$m\ge 2$に対して
        \begin{equation}
            \sum_{n=0}^{m^2}(\sqrt{n}-\qty[\sqrt{n}])
            \ge
            \sum_{l=2}^{m}(\sqrt{l^2-1}-\qty[\sqrt{l^2-1}])
        \end{equation}
        である.

        \begin{equation}
            \qty[\sqrt{l^2-1}]=l-1
        \end{equation}
        を示す.
        $\sqrt{x}$は$x$の単調増加関数であり,$l\ge 2$に対し
        \begin{equation}
            (l-1)^2=l^2-2l+1\le l^2-1<l^2
        \end{equation}
        であるから
        \begin{equation}
            l-1\le \sqrt{l^2-1}<l
        \end{equation}
        \begin{equation}
            \qty[\sqrt{l^2-1}]=l-1
        \end{equation}
        である.

        \begin{equation}
            \sqrt{l^2-1}-\qty[\sqrt{l^2-1}]>\frac{2}{3}
        \end{equation}
        を示す.

        \begin{align}
            \sqrt{l^2-1}-\qty[\sqrt{l^2-1}]
            &=\sqrt{l^2-1}-(l-1)\\
            &=\frac{(l^2-1)-(l-1)^2}{\sqrt{l^2-1}+(l-1)}
            =\frac{2(l-1)}{\sqrt{l^2-1}+(l-1)}\\
            &=\frac{2}{\sqrt{\dfrac{l+1}{l-1}}+1}\\
            &\ge\frac{2}{\sqrt{\dfrac{(l+1)+2(l-2)}{l-1}}+1}
            =\frac{2}{\sqrt{3}+1}\\
            &>\frac{2}{2+1}
            =\frac{2}{3}\\
        \end{align}
        よって示せた.
        以上から,
        \begin{equation}
            \sum_{n=0}^{m^2}(\sqrt{n}-\qty[\sqrt{n}])
            \ge
            \sum_{l=2}^{m}(\sqrt{l^2-1}-\qty[\sqrt{l^2-1}])
            >\dfrac{2(m-1)}{3}
        \end{equation}
        が従う.$m\to\infty$とすると,右辺は正の無限大に発散するから,追い出しの原理により
        \begin{equation}
            \sum_{n=0}^{\infty}(\sqrt{n}-\qty[\sqrt{n}])=\infty \qq{(発散)}
        \end{equation}
        であるから,$h(1)$は発散する.

        \item $h(x)$の収束半径\\
        $h(x)$が$\abs{x}<1$で収束することを優級数法によって示す.\\
        $\abs{x}<1$であるとする.
        \begin{align}
            \abs{\sqrt{n}-\qty[\sqrt{n}]}
            &=\sqrt{n}-\qty[\sqrt{n}]\\
            &<1
        \end{align}
        であることから,べき級数$i(x)=\displaystyle\sum_{n=0}^\infty x^n$について
        \begin{equation}
            \forall n\in \mathbb{Z}_{\ge 0} \quad (\abs{(\sqrt{n}-\qty[\sqrt{n}])x^n}\le x^n)
        \end{equation}
        が成り立ち,$i(x)$は収束することから優級数法により$h(x)$も収束する.$h(x)$の収束半径は$1$である.
    \end{itemize}
\end{enumerate}


\end{document}