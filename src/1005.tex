%& -shell-escape -enable-write18 
\documentclass[uplatex,dvipdfmx]{jsarticle}
\usepackage[utf8]{inputenc}

\usepackage[dvipdfmx]{graphicx}
\usepackage[dvipdfmx]{color}
\usepackage[driver=dvipdfm,hmargin=19.05truemm,vmargin=25.40truemm]{geometry}
\usepackage{colortbl}
\usepackage{tcolorbox}
\usepackage{varwidth}
\usepackage{xcolor}
\PassOptionsToPackage{dvipsnames}{xcolor}

\usepackage{amsmath,amsfonts,amssymb,amsthm}
\usepackage{bm}
\usepackage[italicdiff]{physics}
\usepackage{mathtools}
\mathtoolsset{showonlyrefs=true}
\usepackage[unicode, dvipdfmx]{hyperref}
\usepackage{pxjahyper}
\hypersetup{
setpagesize=false,
    bookmarksnumbered=true,
    bookmarksopen=true,
    colorlinks=true,
    linkcolor=cyan,
    citecolor=red,
}
\usepackage{mathrsfs}
\usepackage{bussproofs}
\usepackage{enumerate}
\usepackage{pxrubrica}
\usepackage{tipa}
\usepackage{ascmac}
\usepackage{caption}
\usepackage[subrefformat=parens]{subcaption}
\usepackage{listings,jvlisting}
\usepackage{tikz}
\usetikzlibrary{math,patterns,intersections,calc,arrows,graphs}
\usepackage{float}
\usepackage{xparse}
\usepackage{url}
\usepackage{fancyhdr}
\usepackage{multicol}
\usepackage{siunitx}
\usepackage{hhline}
\usepackage{bxbase}
\usepackage[mark=***]{sectionbreak}
\usepackage[geometry]{ifsym}
\usepackage[prefernoncjk]{pxcjkcat}
\usepackage[LGR,T2A,T3,T1]{fontenc}
\usepackage[greek,latin,english,russian,japanese]{pxbabel}

\allowdisplaybreaks


\theoremstyle{definition}
\newtheorem{theorem}{Thm}
\newtheorem{corollary}{Col}
\newtheorem{lemma}{Lem}
\newtheorem{definition}{Def}
\newtheorem{proposition}{Prop}

\tcbuselibrary{most}

\renewcommand{\labelitemi}{$\circ$}
\renewcommand{\labelitemii}{$\triangleright$}

\ProvideDocumentCommand\floor{m}{\lfloor {#1} \rfloor}
\ProvideDocumentCommand{\where}{}{\mathrel{}\middle|\mathrel{}}
\ProvideDocumentCommand{\when}{m}{\quad({#1})}
\ProvideDocumentCommand{\adjoint}{}{\mathbf{\ast}}
\ProvideDocumentCommand{\conjugation}{}{\mathbf{\ast}}
\NewDocumentCommand{\Jacobi}{m m}{\frac{\partial ({#1})}{\partial ({#2})}}
\NewDocumentCommand{\JacobiPolar}{O{x,y}}{\frac{\partial ({#1})}{\partial (r,\theta)}}
\DeclareMathOperator{\Ker}{Ker}
\DeclareMathOperator{\Img}{Im}
\DeclareMathOperator{\res}{Res}
\DeclareMathOperator{\Dom}{Dom}
\DeclareMathOperator{\Ran}{Ran}

\DeclareMathOperator{\exd}{d}

\DeclareFontShape{JY2}{mc}{m}{it}{<->ssub*mc/m/n}{}
\DeclareFontShape{JY2}{mc}{m}{sl}{<->ssub*mc/m/n}{}
\DeclareFontShape{JY2}{mc}{m}{sc}{<->ssub*mc/m/n}{}
\DeclareFontShape{JY2}{gt}{m}{it}{<->ssub*gt/m/n}{}
\DeclareFontShape{JY2}{gt}{m}{sl}{<->ssub*gt/m/n}{}
\DeclareFontShape{JY2}{mc}{b}{it}{<->ssub*mc/bx/it}{}
\DeclareFontShape{JY2}{mc}{bx}{it}{<->ssub*gt/m/n}{}
\DeclareFontShape{JY2}{mc}{bx}{sl}{<->ssub*gt/m/n}{}
\DeclareFontShape{JT2}{mc}{m}{it}{<->ssub*mc/m/n}{}
\DeclareFontShape{JT2}{mc}{m}{sl}{<->ssub*mc/m/n}{}
\DeclareFontShape{JT2}{mc}{m}{sc}{<->ssub*mc/m/n}{}
\DeclareFontShape{JT2}{gt}{m}{it}{<->ssub*gt/m/n}{}
\DeclareFontShape{JT2}{gt}{m}{sl}{<->ssub*gt/m/n}{}
\DeclareFontShape{JT2}{mc}{b}{it}{<->ssub*gt/b/n}{}
\DeclareFontShape{JT2}{mc}{bx}{it}{<->ssub*gt/m/n}{}
\DeclareFontShape{JT2}{mc}{bx}{sl}{<->ssub*gt/m/n}{}

\DeclareFontShape{J30}{mc}{b}{it}{<->ssub*mc/bx/it}{}
\DeclareFontShape{J30}{mc}{b}{n}{<->ssub*mc/bx/n}{}
\DeclareFontShape{J20}{mc}{b}{it}{<->ssub*mc/bx/it}{}
\DeclareFontShape{J20}{mc}{b}{n}{<->ssub*mc/bx/n}{}
\ProvideDocumentCommand{\titleof}{m}{

\title{ここに講義タイトルを入力 \\ {#1}レポート課題}
\author{ここに名前を入力}
\date{}
}
\titleof{10/05}

\begin{document}
    
\maketitle

\begin{enumerate}[(1)]
    \item 
    \begin{equation}
        f(x):=\frac{1}{1+x}=\sum_{n=0}^{\infty} (-1)^nx^n
    \end{equation}
    の収束半径は$1$であるから,$\abs{x}<1$のもとで
    \begin{align}
        \int_0^x f(t)\dd{t}
        &=\sum_{n=0}^{\infty} (-1)^n\frac{x^{n+1}}{n+1}\\
        \log(1+x)
        &=\sum_{n=1}^{\infty} (-1)^{n-1}\frac{x^{n}}{n}\label{eq:log_1plusx}
    \end{align}
    である.また,
    \begin{align}
        \dv{x}(f(x))
        &=\sum_{n=1}^{\infty} (-1)^{n-1}nx^{n-1}\\
        &=\sum_{n=0}^{\infty} (-1)^{n}(n+1)x^{n}
    \end{align}
    である.\eqref{eq:log_1plusx}式より
    \begin{align}
        \int_0^x \log(1+t)\dd{t}
        &=\sum_{n=1}^{\infty} (-1)^{n-1}\frac{x^{n+1}}{n(n+1)}\\
        (1+x)\log(1+x)-x
        &=\sum_{n=2}^{\infty} (-1)^{n-2}\frac{x^{n}}{n(n-1)}\\
        (1+x)\log(1+x)
        &=x+\sum_{n=2}^{\infty} (-1)^{n}\frac{x^{n}}{n(n-1)}
    \end{align}
    である.
    
    \item 
    求める$r$の範囲は$0<r<1$であることを示す.
    \begin{proof}
        \begin{itemize}
            \item $0<r<1$のとき\\
            ある数列$\{a_n\}$が
            \begin{equation}
                \abs{a_{n+1}-a_n}<r^n \quad (n =0,1,2,\dots)
            \end{equation}
            を満たしているとする.\\
            $\varepsilon>0$をとる.自然数$l,m$が$l>m>R:=\dfrac{\log((1-r)\min\{\varepsilon,1\})}{\log r}$を満たしているとき,
            \begin{align}
                \abs{a_l-a_m}
                &\le\sum_{n=m}^{l-1}\abs{a_{n+1}-a_n}&(\because\text{三角不等式})\\
                &<\sum_{n=m}^{l-1}r^n\\
                &=r^m\frac{1-r^{l-m}}{1-r}\\
                &<r^m\frac{1}{1-r}&(\because l-m>0)\\
                &<(1-r)\min\{\varepsilon,1\}\frac{1}{1-r}&(\because m>R)\\
                &=\min\{\varepsilon,1\}\\
                \therefore \abs{a_l-a_m}
                &<\varepsilon
            \end{align}
            である.
            したがって,
            たとえば$M:=\floor{R}$とすれば
            任意の自然数$l>m(>M)$に対して
            \begin{equation}
                \abs{a_l-a_m}<\varepsilon
            \end{equation}
            であるから,数列$\{a_n\}$は収束する.
            \item $r\ge 1$のとき\\
            \begin{equation}
                \abs{a_{n+1}-a_n}<1\le r^n \ \ (n =0,1,2,\dots)
            \end{equation}
            を満たす数列$\{a_n\}$として,
            \begin{equation}
                \{a_n\}=\frac{1}{2}n
            \end{equation}
            があるが,これは収束しない.
        \end{itemize}
        以上から,求める$r$の範囲は$0<r<1$である.
    \end{proof}
        \item 
        \begin{align}
            f(x)&=\sum_{n=0}^\infty a_nx^n&(\abs{x}<r),\\
            g(x)&=\sum_{n=1}^\infty na_nx^{n-1}&(\abs{x}<r^\prime)
        \end{align}
        とする.
        \begin{enumerate}[(i)]
            \item 
            \begin{itemize}
                \item $\abs{x}<r$のとき,
                \begin{align}
                    \dv{x}(f(x))=\sum_{n=1}^\infty na_nx^{n-1}=g(x)
                \end{align}
                であるから,$g(x)$は収束し$r\le r^\prime$である.
                \item $\abs{x}<r^\prime$のとき,
                \begin{align}
                    \int_{0}^xg(t)\dd{t}=\sum_{n=1}^\infty a_nx^{n}=f(x)-a_0
                \end{align}
                であるから,$f(x)$は収束し$r\ge r^\prime$である.
                以上から,$r=r^\prime$である.\qed
            \end{itemize}
        \item 
        \begin{align}
            \lim_{n\to\infty}b_n&=\beta>0\\
            \limsup_{n\to\infty}c_n&=\gamma>0
        \end{align}
        が満たされているとする.また
        \begin{equation}
            S_m:=\{c_m, c_{m+1},c_{m+2},\dots\}
        \end{equation}
        とする.

        $\varepsilon>0$をとる.
        \begin{equation}
            \lim_{n\to\infty}b_n=\beta
        \end{equation}
        が成り立っているので,ある$N_1\in\mathbb{N}$が存在して,$\varepsilon_1=\min\{\beta,\dfrac{\varepsilon}{\gamma}\}$に対して
        \begin{align}
            n>N_1&\to \abs{b_n - \beta}<\varepsilon_1\\
            n>N_1&\to 0<\beta - \varepsilon_1 <b_n <\beta + \varepsilon_1\label{eq:3_2_b}
        \end{align}
        が成り立つ.

        \begin{equation}
            \limsup_{n\to\infty}c_n=\gamma
        \end{equation}
        が成り立っているので,
        \begin{equation}
            \lim_{n\to\infty}S_n=\gamma
        \end{equation}
        であり,ある$N_2\in\mathbb{N}$が存在して,$\varepsilon_2=\dfrac{\varepsilon-\gamma\varepsilon_1}{\beta+\varepsilon_1}>0$に対して
        \begin{align}
            \forall m>N_2&\to S_m<\gamma+\varepsilon_2\\
            \forall n>N_2&\to c_n<\gamma+\varepsilon_2\label{eq:3_2_c}
        \end{align}
        が成り立つ.さらに,$\varepsilon_3=\dfrac{\varepsilon-\gamma\varepsilon_1}{\beta-\varepsilon_1}>0$に対して
        \begin{equation}
            c_n>\gamma-\varepsilon_3\label{eq:3_2_c_3}
        \end{equation}
        となる$n$の個数は無限である.

        $n>N=\max\{N_1,N_2\}$とすると,$b_n>0$であり,式\eqref{eq:3_2_b},\eqref{eq:3_2_c}により
        \begin{align}
            b_nc_n 
            &< (\beta + \varepsilon_1)(\gamma+\varepsilon_2)\\
            &= \beta\gamma + \varepsilon_2(\beta+\varepsilon_1)+\gamma\varepsilon_1\\
            &= \beta\gamma + \varepsilon-\gamma\varepsilon_1+\gamma\varepsilon_1\\
            &= \beta\gamma + \varepsilon\\
            \therefore b_nc_n
            &< \beta\gamma + \varepsilon
        \end{align}
        である.よって,$b_nc_n \ge\beta\gamma + \varepsilon$となる$n$の個数は高々$N+1$個であり,有限である.

        また,式\eqref{eq:3_2_c_3}を満たすような$n$であって$n>N_1$も満たすものの個数も無限であり,このような$n$に対し,式\eqref{eq:3_2_b}より
        \begin{align}
            b_nc_n 
            &> (\beta-\varepsilon_1)(\gamma-\varepsilon_3)\\
            &=\beta\gamma - \varepsilon_3(\beta-\varepsilon_1)-\gamma\varepsilon_1\\
            &=\beta\gamma - (\varepsilon-\gamma\varepsilon_1)-\gamma\varepsilon_1\\
            &=\beta\gamma - \varepsilon\\
            \therefore b_nc_n 
            &> \beta\gamma - \varepsilon
        \end{align}
        が成り立つ.

        以上から,
        \begin{align}
            \limsup_{n\to\infty}b_nc_n=\beta\gamma
        \end{align}
        である.\qed
        \item 
        \begin{equation}
            h(x)\coloneqq\sum_{n=1}^\infty na_nx^{n}
        \end{equation}
        としたとき,
        \begin{equation}
            g(x)=\frac{1}{x}h(x) \quad (0<\abs{x}<r^\prime)
        \end{equation}
        である.$h(x)$の収束半径を$r^{\prime\prime}$とする.また,
        \begin{equation}
            \limsup_{n\to\infty}\sqrt[n]{\abs{a_n}}=\gamma
        \end{equation}
        とおく.
        Cauchy-Hadamardの公式により,
        \begin{equation}
            r=\frac{1}{\gamma}
        \end{equation}
        である.
        \begin{equation}
            \limsup_{n\to\infty}\sqrt[n]{\abs{na_n}}=\gamma
        \end{equation}
        であることを示す.
        
        \begin{proof}
            \begin{itemize}
                \item $\gamma=0$のとき\\
                次の補題を示す.
                \begin{lemma}\label{lem:limsup_lim}
                    数列$\qty{a_n}$が
                    \begin{align}
                        \limsup_{n\to\infty}a_n&=\alpha\ (\in\mathbb{R})\\
                        a_n&\ge \alpha\ (\forall n\in\mathbb{N})
                    \end{align}
                    を満たしているとき,
                    \begin{equation}
                        \lim_{n\to\infty}a_n=\alpha
                    \end{equation}
                    である.
                \end{lemma}
                \begin{proof}
                    \begin{equation}
                        \limsup_{n\to\infty}a_n=\alpha
                    \end{equation}
                    であるとする.
                    \begin{equation}
                        S_m:=\sup\qty{a_m,a_{m+1},\dots}
                    \end{equation}
                    を考えると,
                    \begin{equation}
                        \lim_{m\to\infty}S_m=\alpha
                    \end{equation}
                    であるから
                    $\varepsilon>0$に対し,$M\in\mathbb{N}$が存在して,
                    \begin{align}
                        \forall m>M &\to S_m<\alpha+\varepsilon\\
                        \therefore
                        \forall n>M &\to a_n<\alpha+\varepsilon \label{eq:3_3_right}
                    \end{align}
                    が成り立つ.また,
                    \begin{equation}
                        a_n\ge \alpha\ (\forall n\in\mathbb{N})
                    \end{equation}
                    であるとすると
                    \begin{equation}
                        \forall n\in\mathbb{N}\to a_n > \alpha- \varepsilon\label{eq:3_3_left}
                    \end{equation}
                    が成り立つ.
                    \eqref{eq:3_3_right},\eqref{eq:3_3_left}式から
                    \begin{align}
                        \forall n>M &\to \alpha- \varepsilon<a_n<\alpha+\varepsilon\\
                        \therefore
                        \lim_{n\to\infty}a_n=\alpha
                    \end{align}
                    である.
                \end{proof}
                
                
                
                数列$\sqrt[n]{n}>0$が正の値に収束することを示す.
                \begin{proof}
                    \begin{align}
                        \lim_{n\to\infty}\log{\sqrt[n]{n}}
                        &=\lim_{n\to\infty}\frac{1}{n}\log n\\
                        &= 0
                    \end{align}
                    であり,指数関数は連続なので
                    \begin{equation}
                        \lim_{n\to\infty}\sqrt[n]{n}=1
                    \end{equation}
                    である.
                \end{proof}
                さて,
                \begin{equation}
                    \sqrt[n]{\abs{a_n}}\ge 0
                \end{equation}
                であることから,補題\ref{lem:limsup_lim}を$a_n:=\sqrt[n]{\abs{a_n}},\alpha:=0$として適用すると
                \begin{equation}
                    \lim_{n\to\infty}\sqrt[n]{\abs{a_n}} = 0
                \end{equation}
                であるから
                \begin{align}
                    \lim_{n\to\infty}\sqrt[n]{\abs{na_n}}
                    &=
                    \qty(\lim_{n\to\infty}\sqrt[n]{n})
                    \qty(\lim_{n\to\infty}\sqrt[n]{\abs{a_n}})\\
                    &=1\cdot 0 = 0\\
                    \therefore
                    \limsup_{n\to\infty}\sqrt[n]{\abs{na_n}}
                    &=0=\gamma
                \end{align}
                である.
                
                \item $\gamma>0$のとき\\
                (ii)で$b_n:=\sqrt[n]{n},\, c_n:=\sqrt[n]{\abs{a_n}}$とすることができ,
                \begin{align}
                    \limsup_{n\to\infty}\sqrt[n]{\abs{na_n}}
                    &=\limsup_{n\to\infty}\sqrt[n]{n}\sqrt[n]{\abs{a_n}}\\
                    &=\qty(\lim_{n\to\infty}\sqrt[n]{n}) \qty(\limsup_{n\to\infty}\sqrt[n]{\abs{a_n}})\\
                    &=\gamma
                \end{align}
                を得る.
            \end{itemize}
        \end{proof}
        Cauchy-Hadamardの公式により
        \begin{equation}
            r^{\prime\prime}=\frac{1}{\gamma}\label{eq:last_1}
        \end{equation}
        であるから
        \begin{equation}
            r=r^{\prime\prime}
        \end{equation}
        である.
        
        $r^\prime=r^{\prime\prime}$を示す.
        \begin{proof}
            \begin{itemize}
                \item $x=0$のとき
                \begin{align}
                    g(0)&=a_1\\
                    h(0)&=0
                \end{align}
                である.
                \item $0<\abs{x}<r^\prime$のとき
                \begin{equation}
                    g(x)=\delta\in\mathbb{R}
                \end{equation}
                であるから,
                \begin{equation}
                    h(x)=x\delta\in\mathbb{R}
                \end{equation}
                である.
                \item $\abs{x}>r^\prime$のとき\\
                べき級数$\displaystyle\sum_{n=1}^\infty na_nx^{n-1}$は収束しないので,その実数倍である$\displaystyle\sum_{n=1}^\infty na_nx^{n}$も収束しない.

                以上から
                \begin{equation}
                    r^\prime=r^{\prime\prime}\label{eq:last_2}
                \end{equation}
                である.
            \end{itemize}
        \end{proof}
        式\eqref{eq:last_1}\eqref{eq:last_2}から
        \begin{equation}
            r=r^{\prime}
        \end{equation}
        である.\qed
    \end{enumerate}
\end{enumerate}

\end{document}