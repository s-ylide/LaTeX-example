%& -shell-escape -enable-write18 
\documentclass[uplatex,dvipdfmx]{jsarticle}
\usepackage[utf8]{inputenc}

\usepackage[dvipdfmx]{graphicx}
\usepackage[dvipdfmx]{color}
\usepackage[driver=dvipdfm,hmargin=19.05truemm,vmargin=25.40truemm]{geometry}
\usepackage{colortbl}
\usepackage{tcolorbox}
\usepackage{varwidth}
\usepackage{xcolor}
\PassOptionsToPackage{dvipsnames}{xcolor}

\usepackage{amsmath,amsfonts,amssymb,amsthm}
\usepackage{bm}
\usepackage[italicdiff]{physics}
\usepackage{mathtools}
\mathtoolsset{showonlyrefs=true}
\usepackage[unicode, dvipdfmx]{hyperref}
\usepackage{pxjahyper}
\hypersetup{
setpagesize=false,
    bookmarksnumbered=true,
    bookmarksopen=true,
    colorlinks=true,
    linkcolor=cyan,
    citecolor=red,
}
\usepackage{mathrsfs}
\usepackage{bussproofs}
\usepackage{enumerate}
\usepackage{pxrubrica}
\usepackage{tipa}
\usepackage{ascmac}
\usepackage{caption}
\usepackage[subrefformat=parens]{subcaption}
\usepackage{listings,jvlisting}
\usepackage{tikz}
\usetikzlibrary{math,patterns,intersections,calc,arrows,graphs}
\usepackage{float}
\usepackage{xparse}
\usepackage{url}
\usepackage{fancyhdr}
\usepackage{multicol}
\usepackage{siunitx}
\usepackage{hhline}
\usepackage{bxbase}
\usepackage[mark=***]{sectionbreak}
\usepackage[geometry]{ifsym}
\usepackage[prefernoncjk]{pxcjkcat}
\usepackage[LGR,T2A,T3,T1]{fontenc}
\usepackage[greek,latin,english,russian,japanese]{pxbabel}

\allowdisplaybreaks


\theoremstyle{definition}
\newtheorem{theorem}{Thm}
\newtheorem{corollary}{Col}
\newtheorem{lemma}{Lem}
\newtheorem{definition}{Def}
\newtheorem{proposition}{Prop}

\tcbuselibrary{most}

\renewcommand{\labelitemi}{$\circ$}
\renewcommand{\labelitemii}{$\triangleright$}

\ProvideDocumentCommand\floor{m}{\lfloor {#1} \rfloor}
\ProvideDocumentCommand{\where}{}{\mathrel{}\middle|\mathrel{}}
\ProvideDocumentCommand{\when}{m}{\quad({#1})}
\ProvideDocumentCommand{\adjoint}{}{\mathbf{\ast}}
\ProvideDocumentCommand{\conjugation}{}{\mathbf{\ast}}
\NewDocumentCommand{\Jacobi}{m m}{\frac{\partial ({#1})}{\partial ({#2})}}
\NewDocumentCommand{\JacobiPolar}{O{x,y}}{\frac{\partial ({#1})}{\partial (r,\theta)}}
\DeclareMathOperator{\Ker}{Ker}
\DeclareMathOperator{\Img}{Im}
\DeclareMathOperator{\res}{Res}
\DeclareMathOperator{\Dom}{Dom}
\DeclareMathOperator{\Ran}{Ran}

\DeclareMathOperator{\exd}{d}

\DeclareFontShape{JY2}{mc}{m}{it}{<->ssub*mc/m/n}{}
\DeclareFontShape{JY2}{mc}{m}{sl}{<->ssub*mc/m/n}{}
\DeclareFontShape{JY2}{mc}{m}{sc}{<->ssub*mc/m/n}{}
\DeclareFontShape{JY2}{gt}{m}{it}{<->ssub*gt/m/n}{}
\DeclareFontShape{JY2}{gt}{m}{sl}{<->ssub*gt/m/n}{}
\DeclareFontShape{JY2}{mc}{b}{it}{<->ssub*mc/bx/it}{}
\DeclareFontShape{JY2}{mc}{bx}{it}{<->ssub*gt/m/n}{}
\DeclareFontShape{JY2}{mc}{bx}{sl}{<->ssub*gt/m/n}{}
\DeclareFontShape{JT2}{mc}{m}{it}{<->ssub*mc/m/n}{}
\DeclareFontShape{JT2}{mc}{m}{sl}{<->ssub*mc/m/n}{}
\DeclareFontShape{JT2}{mc}{m}{sc}{<->ssub*mc/m/n}{}
\DeclareFontShape{JT2}{gt}{m}{it}{<->ssub*gt/m/n}{}
\DeclareFontShape{JT2}{gt}{m}{sl}{<->ssub*gt/m/n}{}
\DeclareFontShape{JT2}{mc}{b}{it}{<->ssub*gt/b/n}{}
\DeclareFontShape{JT2}{mc}{bx}{it}{<->ssub*gt/m/n}{}
\DeclareFontShape{JT2}{mc}{bx}{sl}{<->ssub*gt/m/n}{}

\DeclareFontShape{J30}{mc}{b}{it}{<->ssub*mc/bx/it}{}
\DeclareFontShape{J30}{mc}{b}{n}{<->ssub*mc/bx/n}{}
\DeclareFontShape{J20}{mc}{b}{it}{<->ssub*mc/bx/it}{}
\DeclareFontShape{J20}{mc}{b}{n}{<->ssub*mc/bx/n}{}
\ProvideDocumentCommand{\titleof}{m}{

\title{ここに講義タイトルを入力 \\ {#1}レポート課題}
\author{ここに名前を入力}
\date{}
}
\titleof{10/26}

\begin{document}
    
\maketitle

\begin{enumerate}[(1)]
    \item Cが線分ABの重心であればよい.
    \begin{align}
        \overline{AC}
        &=\frac{\int_0^1xe^x\dd{x}}{\int_0^1e^x\dd{x}}\\
        &=\frac{\eval[(x-1)e^x|_{0}^{1}}{\eval[e^x|_{0}^{1}}\\
        \therefore
        \overline{AC}
        &=\frac{1}{e-1}
    \end{align}
    \item 両辺を$s$で偏微分して,
    \begin{align}
        \pdv{s}\int_s^{st}f(x)\dd{x}&=\pdv{s}\int_1^{t}f(x)\dd{x}\\
        tf(st)-f(s)&=0
    \end{align}
    $t=\dfrac{1}{s}$を代入して
    \begin{align}
        \frac{1}{s}f(1)-f(s)&=0\\
        f(s)&=\frac{1}{s}f(1)
    \end{align}
    これが任意の$s>0$について成り立つから,
    \begin{equation}
        f(x)=\frac{a}{x} \quad (\exists a\in\mathbb{R})
    \end{equation}
    である必要がある.逆に,$f(x)=\dfrac{a}{x}$のとき
    \begin{align}
        \int_s^{st}f(x)\dd{x}
        &=a(\log(st)-\log s)\\
        &=a\log t\\
        &=\int_1^{t}f(x)\dd{x}\\
    \end{align}
    であるから,求める$f$は
    \begin{equation}
        f(x)=\frac{a}{x} \quad (\exists a\in\mathbb{R})
    \end{equation}
    である.
    \item 
    \begin{enumerate}[(i)]
        \item 概形は以下の5つの場合に大別できる.一様連続性の観点から(b),(c),(d)は同じ場合である.
        \begin{figure}[H]
            \begin{minipage}[t]{.33\textwidth}
                \centering
                \begin{tikzpicture}[domain=0:3]
                    \draw[very thin,color=gray] (-0.1,-0.1) grid (2.9,2.9);
                    \draw[->] (-0.3,0) -- (3.3,0) node[right] {$x$};
                    \draw[->] (0,-0.3) -- (0,3.3) node[above] {$y$};
                    
                    \draw (0,0) node[below left] {$\mathrm{O}$};
                    \draw (1,0) node[below] {$1$};
                    \draw (0,1) node[left] {$1$};

                    \draw[domain=0:1.8,color=red] plot function{x*x} node[above right] {$y=f_s(x)$};
                    \filldraw[draw=red, thick, fill=white] (0,0) circle (2pt);
                \end{tikzpicture}
                \caption{(a):$s>1$のとき}
                \label{graph:a}
            \end{minipage}
            \begin{minipage}[t]{.33\textwidth}
                \centering
                \begin{tikzpicture}[domain=0:3]
                    \draw[very thin,color=gray] (-0.1,-0.1) grid (2.9,2.9);
                    \draw[->] (-0.3,0) -- (3.3,0) node[right] {$x$};
                    \draw[->] (0,-0.3) -- (0,3.3) node[above] {$y$};
                    
                    \draw (0,0) node[below left] {$\mathrm{O}$};
                    \draw (1,0) node[below] {$1$};
                    \draw (0,1) node[left] {$1$};

                    \draw[domain=0:3,color=orange] plot function{x} node[above right] {$y=f_s(x)$};
                    \filldraw[draw=orange, thick, fill=white] (0,0) circle (2pt);
                \end{tikzpicture}
                \caption{(b):$s=1$のとき}
                \label{graph:b}
            \end{minipage}
            \begin{minipage}[t]{.33\textwidth}
                \centering
                \begin{tikzpicture}[domain=0:3]
                    \draw[very thin,color=gray] (-0.1,-0.1) grid (2.9,2.9);
                    \draw[->] (-0.3,0) -- (3.3,0) node[right] {$x$};
                    \draw[->] (0,-0.3) -- (0,3.3) node[above] {$y$};
                    
                    \draw (0,0) node[below left] {$\mathrm{O}$};
                    \draw (1,0) node[below] {$1$};
                    \draw (0,1) node[left] {$1$};

                    \draw[domain=0:3,color=green] plot function{sqrt(x)} node[above right] {$y=f_s(x)$};
                    \filldraw[draw=green, thick, fill=white] (0,0) circle (2pt);
                \end{tikzpicture}
                \caption{(c):$0<s<1$のとき}
                \label{graph:c}
            \end{minipage}
        \end{figure}
        \begin{figure}[H]
            \begin{minipage}[t]{.45\textwidth}
                \centering
                \begin{tikzpicture}[domain=0:3]
                    \draw[very thin,color=gray] (-0.1,-0.1) grid (2.9,2.9);
                    \draw[->] (-0.3,0) -- (3.3,0) node[right] {$x$};
                    \draw[->] (0,-0.3) -- (0,3.3) node[above] {$y$};
                    
                    \draw (0,0) node[below left] {$\mathrm{O}$};
                    \draw (0,1) node[left] {$1$};

                    \draw[domain=0:3,color=cyan] plot function{1} node[above right] {$y=f_s(x)$};
                    \filldraw[draw=cyan, thick, fill=white] (0,1) circle (2pt);
                \end{tikzpicture}
                \caption{(d):$s=0$のとき}
                \label{graph:d}
            \end{minipage}
            \begin{minipage}[t]{.45\textwidth}
                \centering
                \begin{tikzpicture}[domain=0:3]
                    \draw[very thin,color=gray] (-0.1,-0.1) grid (2.9,2.9);
                    \draw[->] (-0.3,0) -- (3.3,0) node[right] {$x$};
                    \draw[->] (0,-0.3) -- (0,3.3) node[above] {$y$};
                    
                    \draw (0,0) node[below left] {$\mathrm{O}$};
                    \draw (1,0) node[below] {$1$};
                    \draw (0,1) node[left] {$1$};

                    \draw[domain=0.3:3,color=blue] plot function{1/x} node[above right] {$y=f_s(x)$};
                \end{tikzpicture}
                \caption{(e):$s<0$のとき}
                \label{graph:e}
            \end{minipage}
        \end{figure}
        \item $0\le s \le 1$である.
        \begin{enumerate}[(a)]
            \item $s>1$のとき$f_s(x)$が一様連続でないことの証明
            \begin{lemma}
                $s>1,x>0$ならば$(1+x)^s>1+sx$である.\label{lem:1xge1psx}
            \end{lemma}
            \begin{proof}
                $s>1$とする.
                $g(x)=(1+x)^s-(1+sx)$とおくと,
                \begin{equation}
                    \dv{g(x)}{x} =s\qty{(1+x)^{s-1}-1}>0\ (x>0)
                \end{equation}
                であり$g(0)=0$だから
                \begin{equation}
                    x>0 \qq{ならば}g(x)>0
                \end{equation}
                すなわち
                \begin{equation}
                    x>0 \qq{ならば}(1+x)^s>1+sx
                \end{equation}
                である.
            \end{proof}
            \begin{proposition}
                $s>1$のとき$f_s(x)=x^s$は一様連続でない.
            \end{proposition}
            \begin{proof}
                $\delta>0$を任意にとる.
                \begin{equation}
                    \begin{dcases}
                        x&=y+\frac{\delta}{s}\ (>0)\\
                        y&=\qty(\frac{1}{\delta})^{\frac{1}{s-1}}\ (>0)
                    \end{dcases}
                \end{equation}
                とすることができる.Lem.\ref{lem:1xge1psx}の両辺を$y^s$倍し$x=\dfrac{\delta}{sy}$を代入して
                \begin{equation}
                    (y+\frac{\delta}{s})^s>y^s+s\cdot\frac{\delta}{sy}\cdot y^s
                \end{equation}
                を得る.
                \begin{align}
                    \abs{x^s-y^s}
                    &=\qty(y+\frac{\delta}{s})^s-y^s\\
                    &>s\cdot\frac{\delta}{sy}\cdot y^s\\
                    &=\delta y^{s-1}\\
                    &=1\\
                    \therefore
                    \abs{f_s(x)-f_s(y)}
                    &>1
                \end{align}
                となるから,$f_s(x)$は一様連続でない.
            \end{proof}
            \item $s=1$のとき$f_1(x)=x$は一様連続であることの証明
            \begin{proposition}
                $s=1$のとき$f_1(x)=x$は一様連続である.
            \end{proposition}
            \begin{proof}
                任意の$\varepsilon>0$に対し,$\delta=\varepsilon$とすることで,任意の$x,y$に対し
                \begin{equation}
                    0<\abs{x-y}<\delta \to \abs{f(x)-f(y)} = \abs{x-y} < \varepsilon
                \end{equation}
                が成り立つ.よって$f_1(x)=x$は一様連続である.
            \end{proof}
            \item $0<s<1$のとき$f_s(x)=x^s$は一様連続であることの証明
            \begin{lemma}
                $0<s<1,x>0$ならば$1+x^s>(1+x)^s$である.\label{lem:unpack_1}
            \end{lemma}
            \begin{proof}
                $0<s<1$とする.
                $h(x)=1+x^s-(1+x)^s$とおくと,
                \begin{equation}
                    \dv{h(x)}{x} =s\qty{x^{s-1}-(1+x)^{s-1}}>0\ (x>0)\ \because \qq{$f_{s-1}(x)$は単調減少関数}
                \end{equation}
                であり$h(0)=0$だから
                \begin{equation}
                    x>0 \qq{ならば}h(x)>0
                \end{equation}
                すなわち
                \begin{equation}
                    x>0 \qq{ならば}1+x^s>(1+x)^s
                \end{equation}
                である.
            \end{proof}
            \begin{proposition}
                $0<s<1$のとき$f_s(x)=x^s$は一様連続である.
            \end{proposition}
            \begin{proof}
                $\delta=\varepsilon^{\frac{1}{s}}$であるとする.
                任意の$x,y$に対し,
                \begin{equation}
                    \abs{x-y}<\delta
                \end{equation}
                であるとする.$x\ge y$のとき
                \begin{equation}
                    x<y+\delta
                \end{equation}
                $f_s(x)$は単調増加関数なので
                \begin{equation}
                    x^s<(y+\delta)^s
                \end{equation}
                であり,Lem.\ref{lem:unpack_1}の両辺を$y^s$倍して$x=\dfrac{\delta}{y}$を代入することで
                \begin{align}
                    x^s<(y+\delta)^s&<y^s+\delta^s\\
                    x^s-y^s&<\delta^s
                \end{align}
                また,$y<x$のときも同様に
                \begin{align}
                    y&<x+\delta\\
                    \therefore 
                    y^s<(x+\delta)^s&< x^s+\delta^s\\
                    y^s-x^s&< \delta^s
                \end{align}
                よって
                \begin{align}
                    \abs{f_s(x)-f_s(y)}&=\abs{x^s-y^s}<\delta^s=\varepsilon\\
                    \therefore 
                    \abs{f_s(x)-f_s(y)}&<\varepsilon
                \end{align}
                が成り立つ.よって$f_s(x)=x^s$は一様連続である.
            \end{proof}
            \item $s=0$のとき$f_0(x)=1$は一様連続であることの証明
            \begin{proposition}
                $s=0$のとき$f_0(x)=1$は一様連続である.
            \end{proposition}
            \begin{proof}
                任意の$x,y$に対し
                \begin{align}
                    \abs{f(x)-f(y)}=0
                \end{align}
                が成り立つ.よって$f_0(x)=1$は一様連続である.
            \end{proof}
            \item $s<0$のとき$f_s(x)=x^s$は一様連続でないことの証明
            \begin{proposition}
                $s<0$のとき$f_s(x)=x^s$は一様連続でない.
            \end{proposition}
            \begin{proof}
                $\delta>0$を任意にとる.$-s>0$であることに注意して,
                \begin{equation}
                    \left\{
                        \begin{aligned}
                            a&=1+\abs{s}^{\frac{1}{s}}\delta\\
                            x&=y-\frac{\delta}{a}\ (>0)\\
                            y&=\abs{\frac{a}{s\delta}}^{\frac{1}{s-1}}\ (>0)
                        \end{aligned}
                    \right.
                \end{equation}
                とすることができる.
                \begin{proof}
                    \begin{equation}
                        x
                        =y-\frac{\delta}{a}
                        >0
                    \end{equation}
                    となる条件は,
                    \begin{align}
                        y=\abs{\frac{a}{s\delta}}^{\frac{1}{s-1}}&>\frac{\delta}{a}\\
                        a^{\frac{s}{s-1}}
                        &>\abs{s\delta}^{\frac{1}{s-1}}\delta\\
                        &=\abs{s}^{\frac{1}{s-1}}\delta^{\frac{s}{s-1}}
                    \end{align}
                    であって$x \mapsto x^{s/(s-1)}$は単調増加だから,$a=1+\abs{s}^{1/s}\delta$はこれを満たす.
                \end{proof}
                また,$a>1$より$\abs{x-y}=\dfrac{\delta}{a}<\delta$である.
                
                $f_s(x)$は下に凸なので,任意の$u>0,v>0$に対して
                \begin{equation}
                    u^s\ge v^s+sv^{s-1}(u-v)
                \end{equation}
                が成り立つから,
                \begin{align}
                    \abs{f(x)-f(y)}
                    &\ge\abs{sy^{s-1}(x-y)}\\
                    &=\abs{s}\abs{\frac{a}{s\delta}}\frac{\delta}{a}\\
                    &=1\\
                    \therefore
                    \abs{f(x)-f(y)}
                    &\ge 1
                \end{align}
                が成り立つ.
                よって$f_0(x)=1$は一様連続でない.
            \end{proof}
        \end{enumerate}
    \end{enumerate}
\end{enumerate}
\end{document}